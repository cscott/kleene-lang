
\documentclass[letterpaper,12pt]{article}
% \documentclass[a4paper,12pt]{article}
% twocolumn letterpaper 10pt 11pt twoside

% for other type sizes, 8, 9, 10, 11, 12, 14pt, 17pt, 20pt
% \documentclass[14pt]{extarticle}
% also extbook, extletter available
% \usepackage{extsizes}

%\usepackage{endnotes}
% then put \theendnotes where you want them

\usepackage{times}
\usepackage{xspace}

\usepackage[outerbars]{changebar}
% supports \cbstart and \cbend, also the changebar environment
% need to run pdflatex multiple times


%\usepackage{alltt}
\usepackage{fancyvrb}  % \begin{Verbatim}[fontsize=\small]
% or [fontsize=\footnotesize]
%\usepackage{upquote}
% affects \verb and verbatim
% to get straight quotes, straight single quote, straight double
% quotes in verbatim environments


%\usepackage{latexsym}  % \LaTeX{} for LaTeX;  \LaTeXe{} for LaTeX2e
%\usepackage{mflogo}    % \MF{}  for METAFONT;  \MP for METAPOST
%\usepackage{url}       % \url{http://www.xrce.xerox.com/people/beesley}
%\usepackage{lscape}    % allows \begin{landscape} ... \end{landscape}

%\usepackage{tipa}
%\include{ipamacros}  % my macros to allow same input for DA and IPA
%\usepackage{desalph}
%\usepackage{arabtex} % see usepackage{buck} and setcode{buck} below
%\usepackage{buck}
%\usepackage{mxedruli}

%\usepackage{epsfig}
%\usepackage{pslatex}  % make whole doc. use postscript fonts

% parallel columns, see also multicol
%\usepackage{parcolumns}
%...
%\begin{parcolumns}[<options>]{3}
%\colchunk{ column 1 text }
%\colchunk{ column 2 text }
%\colchunk{ column 3 text }
%\colplacechunks
%...
%\end{parcolumns}


% for more of these names, see Guide to LaTeX, p. 351
%\providecommand*{\abstractname}{}     % in case the style defines one
%\renewcommand*{\abstractname}{Transcriber notes}
%\renewcommand*{\figurename}{Figure}
%\renewcommand*{\tablename}{Table}
%\renewcommand*{\bibname}{Bibliography}
%\renewcommand*{\refname}{References}

\providecommand{\acro}{}\renewcommand{\acro}{\textsc}
\providecommand{\defin}{}\renewcommand{\defin}{\textsc}

\newcommand{\xmlelmt}{\texttt}
\newcommand{\xmlattr}{\texttt}
\newcommand{\key}{\textbf}
\newcommand{\translit}{\texttt}

% forced pagebreak
%\newpage

%\usepackage{ulem}
%    \uline{important}   underlined text
%    \uuline{urgent}     double-underlined text
%    \uwave{boat}        wavy underline
%    \sout{wrong}        line drawn through word (cross out, strike out)
%    \xout{removed}      marked over with //////.
%    {\em phasized\/}  | In LaTeX, by default, these are underlined; use
%    \emph{asized}     | \normalem or [normalem] to restore italics
%    \useunder{\uwave}{\bfseries}{\textbf}
%                        use wavy underline in place of bold face


%                        \usepackage{natbib}
%\usepackage[authoryear]{natbib}
% compatible with \bibliographystyle{plain}, harvard, apalike, chicago, astron, authordate

%\citet for "textual"   \citet{jon90} ->  Jones et al. (1990)
%\citet[before][after]{key} e.g. \citet[see][p.~47]{jon90} --> 
%         see Jones et al.(1990, chap. 2)
%\citet[chap. 2]{jon90}	    -->    	Jones et al. (1990, chap. 2)
%\citet[after]{key}

%   citep for "parenthetical"
%\citep{jon90}	    -->    	(Jones et al., 1990)
%\citep[chap. 2]{jon90}	    -->    	(Jones et al., 1990, chap. 2)
%\citep[see][]{jon90}	    -->    	(see Jones et al., 1990)
%\citep[see][chap. 2]{jon90}	    -->    	(see Jones et al., 1990, chap. 2)

%\citep for "parenthetical" (author's name in parens)
%\citep  similar
%
%\citet*{key}  list all authors, not just et.al
%\citetext{priv.\ comm.} comes out as (priv. comm.)
%
%just the author or year
%\citeauthor{key} comes out as "Jones et al."
%\citeauthor*{key} comes out as "Jones, Sacco and Vanzetti"
%\citeyear{key}   comes out as 1990
%\citeyearpar{key}            (1990)
%
%Rare stuff:
%use \Citet and \Citep for exceptional forcing of initcap on names
%like 'della Robbia' when it appears first in a sentence.
%
%\citealt like \citet but without parens
%\citealp like \citep but without parens
%


% fancyheadings from The Book (old, obsolete, I think)
%\usepackage{fancyheadings}
%\pagestyle{fancyplain}
% remember the chapter title
%\renewcommand{\chaptermark}[1]{\markboth{#1}{}}
%\renewcommand{\sectionmark}[1]{\markright{\thesection\ #1}}
%\lhead[\fancyplain{}{\small\scshape\thepage}]{\fancyplain{}{\small\scshape\rightmark}}
%\rhead[\fancyplain{}{\small\scshape\leftmark}]{\fancyplain{}{\small\scshape\thepage}}
%\cfoot{}

% new fancyhdr package
%\usepackage{fancyhdr}
%\pagestyle{fancy}
%\fancyhead{}

%% L/C/R denote left/center/right header (or footer) elements
%% E/O denote even/odd pages

%% \leftmark, \rightmark are chapter/section headings generated by the 
%% book document class

%\fancyhead[LE,RO]{\slshape\thepage}
%\fancyhead[RE]{\slshape \leftmark}
%\fancyhead[LO]{\slshape \rightmark}
%\fancyfoot[LO,LE]{\slshape Short Course on Asymptotics}
%\fancyfoot[C]{}
%\fancyfoot[RO,RE]{\slshape 7/15/2002}

% another example
%\fancyhead[LE]{\thepage}
%\fancyhead[CE]{\bfseries Beesley}
%\fancyfoot[CE]{First Draft}
%\fancyhead[CO]{\bfseries My Article Title}
%\fancyhead[RO]{\thepage}
%\fancyfoot[CO]{For Review and Editing Only}
%\renewcommand{\footrulewidth}{0.4pt}

% \vspace{.5cm}
% c, l, r, p{1cm}
%\begin{tabular}{}
%\hline
%   &  &  &   \\
%\hline
%\end{tabular}
% \vspace{.5cm}


% bigbox -- puts a box around a float
% for {figure}, {table} or {center}

\newdimen\boxfigwidth  % width of figure box

\def\bigbox{\begingroup
  % Figure out how wide to set the box in
  \boxfigwidth=\hsize
  \advance\boxfigwidth by -2\fboxrule
  \advance\boxfigwidth by -2\fboxsep
  \setbox4=\vbox\bgroup\hsize\boxfigwidth
  % Make an invisible hrule so that
  % the box is exactly this wide
  \hrule height0pt width\boxfigwidth\smallskip%
% Some environments like TABBING and other LIST environments
% use this measure of line size -
% \LINEWIDTH=\HSIZE-\LEFTMARGIN-\RIGHTMARGIN?
  \linewidth=\boxfigwidth
}
\def\endbigbox{\smallskip\egroup\fbox{\box4}\endgroup}


% example
% \begin{figure}
%   \begin{bigbox}
%     \begin{whatever}...\end{whatever}
%     \caption{}
%     \label{}
%   \end{bigbox}
% \end{figure}
% 
% N.B. put the caption and label inside the bigbox

%\usepackage{graphicx}
% Sample Graphics inclusion; needs graphicx package
%\begin{figure}[ht]
%\begin{bigbox}
%\centering
%\includegraphics{foobar.pdf}   # e.g. PNG, PDF or JPG, _not_ EPS
%\caption{}
%\label{lab:XXX}
%\end{bigbox}
%\end{figure}

%\pagestyle{empty}  % to suppress page numbering

% turn text upside down
%\reflectbox{\textipa{\textlhookp}}
% prevent line break:   \mbox{...}

\hyphenation{hy-po-cri-tical ri-bald}

%%%%%%%%%%%%%%%%%%%%  title %%%%%%%%%%%%%%%%%%%%%%%%%%%%%%

\title{OpenFstRestrictions:\\
Collecting and Refactoring Compatibility Checks into a New Class}
\author{Kenneth R.~Beesley}

% to override automatic "today" date
\date{15 October 2010}

%%%%%%%%%%%%%%%%%%%%%% document %%%%%%%%%%%%%%%%%%%%%%%%%%

\begin{document}
\maketitle

\section{Introduction}

It is proposed to collect and refactor the \acro{fst} compatibility checks,
currently scattered through the Kleene interpreter
class OpenFstInterpreterKleeneVisitor.java, into a new class, to be called
something like OpenFstRestrictions.java.  Compatibility checks determine
if a network can undergo a particular operation, such as Determinize(), or if two 
networks can be legally and ``safely'' combined using a
particular binary OpenFst operation, e.g.
cross-product, intersection or difference.

It is anticipated that this will have a number of benefits:

\begin{enumerate}
\item
Simplify the reading and editing of the interpreter
\item
Reduce the size of the interpreter class, which is by far the largest file in the Kleene
code-base
\item
Facilitate refactoring of the compatibility checks
\item
Facilitate the updating of compatibility checks when the OpenFst-library algorithms
evolve
\item
Facilitate the adding of new checks to handle innovations such as \acro{rtn}s and
(in the future) multiple semirings\footnote{Currently Kleene handles only the
Tropical Semiring, which is the default semiring in OpenFst.}
\end{enumerate}

\section{Compatibility Checks}

In the current interpreter, the compatibility checks are included inside the
various methods implementing finite-state operations.  For example, using the
OpenFst Difference() operation, which computes the difference/subtraction of one
\acro{fst} from another \acro{fst}, the following restrictions---specified in the OpenFst
documentation---apply:

\begin{quote}
The first argument must be an acceptor; the second argument must be an unweighted,
epsilon-free, deterministic acceptor.
\end{quote}

\noindent
Inside the interpreter, these restrictions are implemented as the following Java
code:

\begin{Verbatim}[fontsize=\small]
    // the first arg must be an acceptor
    if (!isAcceptor(firstFst.getFstPtr())) {
        throw new FstPropertyException(
        "The first argument to an fst difference must be an Acceptor.") ;
    }

    // the second arg must be 1) acceptor, 2) unweighted, 
    // 3) epsilonFree, and 4) deterministic
    if (!isAcceptor(secondFst.getFstPtr())) {
        throw new FstPropertyException(
        "The second argument to an fst difference must be an Acceptor.") ;
    }
    if (!isUnweighted(secondFst.getFstPtr())) {
        throw new FstPropertyException(
        "The second argument to an fst difference must be unweighted.") ;
    }
    if (!isEpsilonFree(secondFst.getFstPtr())) {
        // epsilon removal done in place on the network
        // (work on a copy?)
        rmEpsilonInPlace(secondFst.getFstPtr()) ;
    }
    if (!isIDeterministic(secondFst.getFstPtr())) {
        // IDeterministic is for "input" deterministic,
        // but this is an acceptor, so IDeterministic
        // and ODeterministic are the same here.

        // It has already been determined that this FST is 
        // (Acceptor and !Weighted)
        // so OpenFst Determinize() (i.e. sequentialize) can be called

        determinizeInPlaceNative(secondFst.getFstPtr()) ;
        // KRB:  possibly call optimizeInPlace() here?
    } 
\end{Verbatim}

The proposal is to take this code, and similar compatibility checks, out of the
interpreter and collect them in a new OpenFstRestrictions class, which would have
methods like the following:

\begin{Verbatim}
OpenFstRestrictions.checkDifference(Fst, Fst) 
OpenFstRestrictions.checkCrossproduct(Fst, Fst)
OpenFstRestrictions.checkIntersect(Fst, Fst)
OpenFstRestrictions.checkComplement(Fst)
OpenFstRestrictions.checkSymbolComplement(Fst)
\end{Verbatim}

\section{Why OpenFstRestrictions?}

There are some standard mathematical restrictions on the \acro{fst}s that can be combined
using various operations; and, in addition, there can be idiosyncratic restrictions
on operations imposed by the algorithms provided in a particular library
implementation of finite-state theory.  For example, the current OpenFst release,
the Determinize() function requires (in addition to some standard mathematical restrictions) 
that the argument transducer also be functional,
and that limitation should be eliminated in a future release.

Kleene is designed to be maximally modular, not tied to a particular library, and
the current interpreter is OpenFstInterpreterKleeneVisitor.java, extending the
KleeneVisitor.java interface, indicating that it is an interpreter based on the OpenFst
library.  The OpenFstRestrictions.java class would encapsulate the compatibility
restrictions appropriate for the OpenFst library.  If an alternative Kleene interpreter is
ever implemented, based on a different library Foo, it would be implemented in
FooInterpreterKleeneVisitor.java and FooRestrictions.java.

More subtle restrictions are linked to the semiring being used.  Kleene currently
supports only the Tropical Semiring, which is useful, well-behaved and the default
in OpenFst.  The OpenFst Determinize() operation also requires that the weights be
``weakly left divisible'', which is the case for the TropicalWeight and the
LogWeight, but not for all possible weights.  When and if Kleene is expanded to
handle multiple semirings, such weight properties will also have to be checked
before some operations can be performed.

The immediate need in Kleene development is to handle \acro{rtn}s, and to prevent any user
attempt to perform illegal operations on \acro{rtn}s, or to combine \acro{rtn} and
non-\acro{rtn}
networks in illegal ways.

By grouping the restriction checking in a separate class, OpenFstRestrictions.java,
the various kinds of present and future restrictions can be better viewed, edited
and augmented.

\section{Marking Networks as \acro{rtn}s}

There is a big question:  How should the code check if a particular \acro{fst} is an
\acro{rtn},
meaning that it contains non-terminal labels that are references to subnetworks?
Actually retrieving the sigma, and looking for non-terminal labels, would be
inefficient.  The better way, at first glance, would be to maintain a boolean rtn field
in \acro{fst}s, accessed with a getRtn() method.\footnote{This would be somewhat similar
to the containsOther() method.}  The challenge then is to set the \acro{fst}'s rtn
field when appropriate, and to change its value appropriately as various operations
are performed on the \acro{fst}.

In the simplest case, the processing of a non-terminal label in a regular
expression would cause the resulting rtn
field to be set to true.  We need to think about cases, e.g.\@ difference and
composition, where non-terminal labels in the arguments might be absent in the
result.


\end{document}
