\chapter{Examples without Weights}

\label{chapt:exampleswithoutweights}

\section{Introduction}

The purpose of this chapter is to present some non-trivial examples that
do not involve weights.  This is definitely work in progress, and
examples will be added, corrected and expanded in future releases.
Example contributions from users would be gratefully received.

\section{Spanish Morphology}

\subsection{Spanish Verbs}

\subsubsection{Mind-Tuning}

I chose Spanish as an initial example because Spanish is widely
studied and extensively documented, and I have some basic
familiarity with it.  I will start with modeling
Spanish verbs---in particular, the highly productive
regular first conjugation.  

The various Spanish verb classes, and
their conjugations, are documented in numerous published
books\footnote{See books such as \emph{501 Spanish Verbs} and especially the
verb-conjugation books published by Larousse and Bescherelle.} and are
even available on the Internet.\footnote{See, for example,
\url{http://www.conjugacion.es} for the conjugations of
Spanish verbs.}  I hope eventually to
offer a Kleene script, downloadable from \url{www.kleene-lang.org},
that handles all Spanish verbs.  I emphasize that my approach in the
following script is only one of many.

By long lexicographic convention, Spanish verbs are listed in
dictionaries under the infinitive form, e.g.\@ the infinitive
\emph{amar}, meaning ``to love,'' is the conventional \emph{dictionary
citation form}.  In Spanish and other languages, the traditional dictionary citation
forms should not be confused with baseforms or roots.  If anything, the
root
of the ``love'' verb is just \emph{am}, to which various suffixes
are attached.  The infinitive citation form, \emph{amar}, is really just \emph{am}
with an \emph{ar} suffix; it is just one of the many conjugated forms
of the verb.  However, in a bow to tradition, we will
initially list verbs in their infinitive forms, and analyses of verbs
will show the infinitive form to facilitate lookup in traditional
dictionaries.  The script will use alternation
rules to strip the infinitive suffixes from the lower side of the
\fsm{}s before adding other suffixes to implement the various conjugated
forms.


\subsubsection{Modeling Spanish First-conjugation Verbs}

A verb class traditionally called the \emph{first conjugation}
contains thousands of regular verbs whose
infinitive forms end in \emph{-ar}, including \emph{amar} ``to love,''
\emph{cantar} ``to sing,'' and \emph{hablar} ``to speak.''  My Larousse
\emph{Conjugación} book
calls this verb class number 3, and I will use the Larousse numbers
(with some modification)
to distinguish the various verb-conjugation classes.  We can start
by collecting a sampling of this class of verbs in a simple union of the
infinitive forms.

\begin{Verbatim}

// "First Conjugation" class 3
$V3CitationForms =
amar |
cantar |
cortar |
hablar |
simular ;	// and continue to add hundreds more
\end{Verbatim}

\noindent
See \url{http://www.www.conjugacion.es/del/verbo/amar.php}, the
Larousse or Bescherelle books, or any of the other sources for a table
showing all the conjugated forms.  We will limit ourselves for now to the
single-word conjugations, ignoring composite multi-word conjugations.
Clitic
pronouns will also be ignored for the time being. 

These charts contain conjugation groups, almost always of six forms,
showing the

\begin{Verbatim}
1st person singular
2nd person singular
3rd person singular
1st person plural
2nd person plural
3rd person plural
\end{Verbatim}

\noindent
forms for present indicative, preterite imperfective, preterite perfect,
future indicative, conditional, etc.  For example, the six present indicative
forms of \emph{amar} are


\begin{Verbatim}
amo
amas
ama
amamos
amáis
aman
\end{Verbatim}

\noindent
showing that \emph{amo} (`I love') is the first-person singular present indicative form of
\emph{amar}, \emph{amas} (`thou (you singular) lovest') is the second-person singular,
\emph{ama} (`he/she loves') is the
third-person singular, \emph{amamos} (`we love') is the first-person plural,
\emph{amáis} (`you (plural) love') is the second-person plural, and \emph{aman} (`they
love') is the third-person plural.   The root
is \emph{am}, and the six
suffixes for this group are pretty obviously \emph{o}, \emph{as}, \emph{a}, \emph{amos},
\emph{áis} and \emph{an}.  We want to
build a morphological analyzer that will eventually accept any 
orthographical verb string, e.g.\@ \emph{amo}, and return
a string that includes the infinitive \emph{amar}, the information that it is a
verb, and the information that it is the first-person singular present indicative
form.  
We will invent and employ a number of multi-character symbol tags to
convey part-of-speech, person, number, tense, aspect and mood information.  Our
\fsm{} will include paths like the following, ignoring alignment and epsilons,

\begin{Verbatim}
Upper:   amar[VERB][V3][PresIndic][1P][Sg]
Lower:   amo
\end{Verbatim}

\noindent
where \texttt{[VERB]}, \texttt{[V3]}, \texttt{[PresIndic]}, \texttt{[1P]} and \texttt{[Sg]} are
multi-character symbols.  Conversely, we want to be able to apply the same
\fsm{} in a downward direction to the upper-side string, and see the output
\emph{amo}.

While the spellings of the lower-side words in the
\fsm{} are determined by the rules of Spanish orthography, the design of the upper-side
analysis strings is in our hands, and some care and study should go into that
design.  This is just one possible design of many.\footnote{Another possibility is to include
\init{xml}-like symbols to mark up the analysis strings.}
Don't worry too much about the spelling of the multi-character symbols;
we will show later that they can be changed trivially at a later time, using
alternation rules.
Multi-characters symbols like \texttt{[V3]}, identifying the Larousse conjugation
class, can also be changed trivially to other tags (e.g.\@ to reflect some other
numbering system) or simply deleted.  Here is the list that we
will use in the script:

\begin{center}
\begin{tabular}{|l|l|}
\hline
[VERB] & verb (part of speech) \\
\hline
[V1], [V2], [V3], [V4], etc. & verb conjugation classes\\
\hline
[PresIndic] & present indicative\\
\hline
[PretImperf] & preterite imperfect\\
\hline
[PretPerf] & preterite perfect\\
\hline
[FutIndic] & future indicative\\
\hline
[Cond] & conditional\\
\hline
[PresSubj] & present subjunctive\\
\hline
[ImperfSubj] & imperfect subjunctive\\
\hline
[Var1] & variant 1 (of the imperfect subjunctive)\\
\hline
[Var2] & variant 2 (of the imperfect subjunctive)\\
\hline
[FutSubj] & future subjunctive\\
\hline
[Imptv] & imperative\\
\hline
[Infin] & infinitive\\
\hline
[PresPart] & present participle\\
\hline
[PastPart] & past participle\\
\hline
\end{tabular}
\end{center}

Starting from the dictionary citation forms, such as \emph{amar}, we want to
leave the full \emph{amar} on the upper side but
effectively strip off the \emph{-ar} infinitive ending on the lower side, leaving
just the root.  This can be accomplished with the following first-draft function,
to be expanded and generalized later:


\begin{Verbatim}
$^stripRegInfinEnding($fst, $classTag) {
    return ( $fst 
            _o_
            ar -> "" / _ # ) ('[VERB]' $classTag):"" ;
}
\end{Verbatim}

\noindent
If we call this function on the word \emph{amar}, with the \$classTag
\verb![V3]!,


\begin{Verbatim}
$stem = $^stripRegInfinEnding(amar, '[V3]') ;
\end{Verbatim}

\noindent
the result is
an \fsm{} with the following path, ignoring alignment and epsilons.

\begin{Verbatim}
Upper:   amar[VERB][V3]
Lower:   am
\end{Verbatim}

Because the conjugations are in groups of six, we can facilitate modeling each
conjugation group by defining the following \verb!$^conj6()! function, which takes seven
arguments, the last six being the six suffixes for a conjugation group.

\begin{Verbatim}
$^conj6($tags, $OnePerSg, $TwoPerSg, $ThreePerSg,
               $OnePerPl, $TwoPerPl, $ThreePerPl) {
    return $tags:"" ( ( '[1P]' '[SG]' ):$OnePerSg
                    | ( '[2P]' '[SG]' ):$TwoPerSg
                    | ( '[3P]' '[SG]' ):$ThreePerSg
                    | ( '[1P]' '[PL]' ):$OnePerPl
                    | ( '[2P]' '[PL]' ):$TwoPerPl
                    | ( '[3P]' '[PL]' ):$ThreePerPl
                    ) ;
}
\end{Verbatim}

\noindent
And we can then use \verb!$^conj6()! to define the regular suffixes for the first-conjugation
verbs including \emph{amar}:

\begin{Verbatim}
$regArVerbSuffs = 		
  ( $^conj6( '[PresIndic]', o, as, a, amos, áis, an )
  | $^conj6( '[PretImperf]', aba, abas, aba,
                             ábamos, abais, aban )
  | $^conj6( '[PretPerf]',  é, aste, ó, 
                            amos, asteis, aron )
  | $^conj6( '[FutIndic]', aré, arás, ará, 
                           aremos, aréis, arán )
  | $^conj6( '[Cond]', aría, arías, aría, 
                       aríamos, aríais, arían )
  | $^conj6( '[PresSubj]',  e, es, e, emos, éis, en ) 
  | $^conj6( '[ImperfSubj]' '[Var1]', ara, aras, ara, 
                                      áramos, arais, aran )
  | $^conj6( '[ImperfSubj]' '[Var2]', ase, ases, ase, 
                                      ásemos, aseis, asen )
  | $^conj6( '[FutSubj]', are, ares, are, 
                          áremos, areis, aren )
  | $^conj6( '[Imptv]', '[Defective]', a, e, emos, ad, en )
  | '[Infin]':(ar)
  | '[PresPart]':(ando)
  | '[PastPart]':(ado)
  ) ;
\end{Verbatim}

The first-person singular present indicative suffix path will look like this:

\begin{Verbatim}
Upper:   [PresIndic][1P][Sg]
Lower:   o
\end{Verbatim}

\noindent
When concatenated with the truncated infinitive, the full path (ignoring alignment and epsilons) looks like

\begin{Verbatim}
Upper:   amar[VERB][V3][PresIndic][1P][Sg]
Lower:   amo
\end{Verbatim}

We can then build our first verb-conjugating \fsm{}, for first-conjugation verbs,  with


\begin{Verbatim}
$V3 = $^stripRegInfinEnding($V3CitationForms, '[V3]')
      $regArVerbSuffs ;

test $V3 ;
\end{Verbatim}


If we apply this \fsm{} in an upward direction to the string \texttt{amo}, meaning
``I love,'' the output is the string \texttt{amar[VERB][V3][PresIndic][1P][Sg]},
which we can read as the citation form \emph{amar}, which is a verb of
class 3, in the present indicative first-person singular conjugation.
And we can do the inverse, applying the same \fsm{} in a downward direction to
the input \texttt{amar[VERB][V3][PresIndic][1P][Sg]}, and the result will be
\texttt{amo}.  The \fsm{} thus \emph{transduces} from strings representing conjugated
verbs to strings representing analyses of those verbs, and vice versa.

Finally, we can define a convenient function, here called \verb!^conj()!, for testing that prints out all the conjugated forms for a
verb, in a canonical order that can be compared to the usual published and
online charts.


\begin{Verbatim}
^conj($fst, $infin) {
    pr("\nPresent Indicative") ;
    pr($^lowerside( ( $infin '[VERB]' . 
                      '[PresIndic]' '[1P]' '[SG]' ) _o_ $fst)) ;
    pr($^lowerside( ( $infin '[VERB]' . 
                      '[PresIndic]' '[2P]' '[SG]' ) _o_ $fst)) ;
    pr($^lowerside( ( $infin '[VERB]' . 
                      '[PresIndic]' '[3P]' '[SG]' ) _o_ $fst)) ;
    pr($^lowerside( ( $infin '[VERB]' . 
                      '[PresIndic]' '[1P]' '[PL]' ) _o_ $fst)) ;
    pr($^lowerside( ( $infin '[VERB]' . 
                      '[PresIndic]' '[2P]' '[PL]' ) _o_ $fst)) ;
    pr($^lowerside( ( $infin '[VERB]' . 
                      '[PresIndic]' '[3P]' '[PL]' ) _o_ $fst)) ;

    pr("\nPreterit Imperfect") ;
    pr($^lowerside( ( $infin '[VERB]' . 
                      '[PretImperf]' '[1P]' '[SG]' ) _o_ $fst)) ;
    pr($^lowerside( ( $infin '[VERB]' . 
                      '[PretImperf]' '[2P]' '[SG]' ) _o_ $fst)) ;
    pr($^lowerside( ( $infin '[VERB]' . 
                      '[PretImperf]' '[3P]' '[SG]' ) _o_ $fst)) ;
    pr($^lowerside( ( $infin '[VERB]' . 
                      '[PretImperf]' '[1P]' '[PL]' ) _o_ $fst)) ;
    pr($^lowerside( ( $infin '[VERB]' . 
                      '[PretImperf]' '[2P]' '[PL]' ) _o_ $fst)) ;
    pr($^lowerside( ( $infin '[VERB]' . 
                      '[PretImperf]' '[3P]' '[PL]' ) _o_ $fst)) ;

    pr("\nPreterit Perfect") ;
    pr($^lowerside( ( $infin '[VERB]' . 
                      '[PretPerf]' '[1P]' '[SG]' ) _o_ $fst)) ;
    pr($^lowerside( ( $infin '[VERB]' . 
                      '[PretPerf]' '[2P]' '[SG]' ) _o_ $fst)) ;
    pr($^lowerside( ( $infin '[VERB]' . 
                      '[PretPerf]' '[3P]' '[SG]' ) _o_ $fst)) ;
    pr($^lowerside( ( $infin '[VERB]' . 
                      '[PretPerf]' '[1P]' '[PL]' ) _o_ $fst)) ;
    pr($^lowerside( ( $infin '[VERB]' . 
                      '[PretPerf]' '[2P]' '[PL]' ) _o_ $fst)) ;
    pr($^lowerside( ( $infin '[VERB]' . 
                      '[PretPerf]' '[3P]' '[PL]' ) _o_ $fst)) ;

    pr("\nFuture Indictive") ;
    pr($^lowerside( ( $infin '[VERB]' . 
                      '[FutIndic]' '[1P]' '[SG]' ) _o_ $fst)) ;
    pr($^lowerside( ( $infin '[VERB]' . 
                      '[FutIndic]' '[2P]' '[SG]' ) _o_ $fst)) ;
    pr($^lowerside( ( $infin '[VERB]' . 
                      '[FutIndic]' '[3P]' '[SG]' ) _o_ $fst)) ;
    pr($^lowerside( ( $infin '[VERB]' . 
                      '[FutIndic]' '[1P]' '[PL]' ) _o_ $fst)) ;
    pr($^lowerside( ( $infin '[VERB]' . 
                      '[FutIndic]' '[2P]' '[PL]' ) _o_ $fst)) ;
    pr($^lowerside( ( $infin '[VERB]' . 
                      '[FutIndic]' '[3P]' '[PL]' ) _o_ $fst)) ;

    pr("\nConditional") ;
    pr($^lowerside( ( $infin '[VERB]' . 
                      '[Cond]' '[1P]' '[SG]' ) _o_ $fst)) ;
    pr($^lowerside( ( $infin '[VERB]' . 
                      '[Cond]' '[2P]' '[SG]' ) _o_ $fst)) ;
    pr($^lowerside( ( $infin '[VERB]' . 
                      '[Cond]' '[3P]' '[SG]' ) _o_ $fst)) ;
    pr($^lowerside( ( $infin '[VERB]' . 
                      '[Cond]' '[1P]' '[PL]' ) _o_ $fst)) ;
    pr($^lowerside( ( $infin '[VERB]' . 
                      '[Cond]' '[2P]' '[PL]' ) _o_ $fst)) ;
    pr($^lowerside( ( $infin '[VERB]' . 
                      '[Cond]' '[3P]' '[PL]' ) _o_ $fst)) ;

    pr("\nPresent Subjunctive") ;
    pr($^lowerside( ( $infin '[VERB]' . 
                      '[PresSubj]' '[1P]' '[SG]' ) _o_ $fst)) ;
    pr($^lowerside( ( $infin '[VERB]' . 
                      '[PresSubj]' '[2P]' '[SG]' ) _o_ $fst)) ;
    pr($^lowerside( ( $infin '[VERB]' . 
                      '[PresSubj]' '[3P]' '[SG]' ) _o_ $fst)) ;
    pr($^lowerside( ( $infin '[VERB]' . 
                      '[PresSubj]' '[1P]' '[PL]' ) _o_ $fst)) ;
    pr($^lowerside( ( $infin '[VERB]' . 
                      '[PresSubj]' '[2P]' '[PL]' ) _o_ $fst)) ;
    pr($^lowerside( ( $infin '[VERB]' . 
                      '[PresSubj]' '[3P]' '[PL]' ) _o_ $fst)) ;

    pr("\nImperfect Subjunctive, Var 1") ;
    pr($^lowerside( ( $infin '[VERB]' . 
                      '[ImperfSubj]' '[Var1]' '[1P]' '[SG]' ) _o_ $fst)) ;
    pr($^lowerside( ( $infin '[VERB]' . 
                      '[ImperfSubj]' '[Var1]' '[2P]' '[SG]' ) _o_ $fst)) ;
    pr($^lowerside( ( $infin '[VERB]' . 
                      '[ImperfSubj]' '[Var1]' '[3P]' '[SG]' ) _o_ $fst)) ;
    pr($^lowerside( ( $infin '[VERB]' . 
                      '[ImperfSubj]' '[Var1]' '[1P]' '[PL]' ) _o_ $fst)) ;
    pr($^lowerside( ( $infin '[VERB]' . 
                      '[ImperfSubj]' '[Var1]' '[2P]' '[PL]' ) _o_ $fst)) ;
    pr($^lowerside( ( $infin '[VERB]' . 
                      '[ImperfSubj]' '[Var1]' '[3P]' '[PL]' ) _o_ $fst)) ;

    pr("\nImperfect Subjunctive, Var 2") ;
    pr($^lowerside( ( $infin '[VERB]' . 
                      '[ImperfSubj]' '[Var2]' '[1P]' '[SG]' ) _o_ $fst)) ;
    pr($^lowerside( ( $infin '[VERB]' . 
                      '[ImperfSubj]' '[Var2]' '[2P]' '[SG]' ) _o_ $fst)) ;
    pr($^lowerside( ( $infin '[VERB]' . 
                      '[ImperfSubj]' '[Var2]' '[3P]' '[SG]' ) _o_ $fst)) ;
    pr($^lowerside( ( $infin '[VERB]' . 
                      '[ImperfSubj]' '[Var2]' '[1P]' '[PL]' ) _o_ $fst)) ;
    pr($^lowerside( ( $infin '[VERB]' . 
                      '[ImperfSubj]' '[Var2]' '[2P]' '[PL]' ) _o_ $fst)) ;
    pr($^lowerside( ( $infin '[VERB]' . 
                      '[ImperfSubj]' '[Var2]' '[3P]' '[PL]' ) _o_ $fst)) ;

    pr("\nFuture Subjunctive") ;
    pr($^lowerside( ( $infin '[VERB]' . 
                      '[FutSubj]' '[1P]' '[SG]' ) _o_ $fst)) ;
    pr($^lowerside( ( $infin '[VERB]' . 
                      '[FutSubj]' '[2P]' '[SG]' ) _o_ $fst)) ;
    pr($^lowerside( ( $infin '[VERB]' . 
                      '[FutSubj]' '[3P]' '[SG]' ) _o_ $fst)) ;
    pr($^lowerside( ( $infin '[VERB]' . 
                      '[FutSubj]' '[1P]' '[PL]' ) _o_ $fst)) ;
    pr($^lowerside( ( $infin '[VERB]' . 
                      '[FutSubj]' '[2P]' '[PL]' ) _o_ $fst)) ;
    pr($^lowerside( ( $infin '[VERB]' . 
                      '[FutSubj]' '[3P]' '[PL]' ) _o_ $fst)) ;

    pr("\nImperative") ;
    pr($^lowerside( ( $infin '[VERB]' . 
                      '[Imptv]' '[2P]' '[SG]' ) _o_ $fst)) ;
    pr($^lowerside( ( $infin '[VERB]' . 
                      '[Imptv]' '[3P]' '[SG]' ) _o_ $fst)) ;
    pr($^lowerside( ( $infin '[VERB]' . 
                      '[Imptv]' '[1P]' '[PL]' ) _o_ $fst)) ;
    pr($^lowerside( ( $infin '[VERB]' . 
                      '[Imptv]' '[2P]' '[PL]' ) _o_ $fst)) ;
    pr($^lowerside( ( $infin '[VERB]' . 
                      '[Imptv]' '[3P]' '[PL]' ) _o_ $fst)) ;

    pr("\nInfinitive") ;
    pr($^lowerside( ( $infin '[VERB]' . 
                      '[Infin]' ) _o_ $fst)) ;

    pr("\nPresent Participle (Gerund)") ;
    pr($^lowerside( ( $infin '[VERB]' . 
                      '[PresPart]' ) _o_ $fst)) ;

    pr("\nPast Participle") ;
    pr($^lowerside( ( $infin '[VERB]' . 
                      '[PastPart]' ) _o_ $fst)) ;
}
\end{Verbatim}

\noindent
One can then simply call \verb!^conj($fst, $infin)! to test various verbs,
printing out the entire conjugation paradigm.  

\begin{Verbatim}
^conj($V3, amar) ;
^conj($V3, cortar) ;
\end{Verbatim}

\noindent
The coverage of our exampler can be expanded trivially to other verbs of the same
first-conjugation class
by simply adding more citation forms to the definition of \verb!$V3CitationForms!.

\subsubsection{Expanding the Spanish Verb Morphological Analyzer}

In addition to the first conjugation, a large class consisting of regular
\mbox{\emph{-ar}} verbs, 
the second conjugation consists of regular \mbox{\emph{-er}} verbs, including \emph{beber} (`to drink'), and the third conjugation consists of
regular \mbox{\emph{-ir}} verbs, including \emph{vivir} (`to live').  Both my Larousse and Bescherelle conjugation guides identify a total
of 90 verb-conjugation classes for Spanish, though their numbering systems differ, and even a
single publisher may not use consistent numbering from one edition to another.  The 90 classes
include a large number of irregular verbs with idiosyncratic conjugations.

The \verb!spanverbs-0.9.4.0.kl! script is offered on the downloads page from
\url{www.kleene-lang.org}.  It is work in progress and incomplete in several
ways:

\begin{enumerate}
	\item
It covers only about 2/3 of the 90 verb-conjugation classes.
\item
The definitions of the various ``CitationForms'' for each class are incomplete,
but could be expanded easily.
\item
The grammar does not yet handle clitic pronouns.
\end{enumerate}

\noindent
The script is in \init{utf}-8 and can be loaded from the Kleene \gui{} by
invoking


\begin{Verbatim}
source "path/to/spanverbs-0.9.4.0.kl", "UTF-8" ;
\end{Verbatim}

\noindent
from the pseudo-terminal, replacing ``path/to'' with the path to where
you have stored the script on your file system.  The final result of the compilation is
\verb!$spanverbs!, and it can be tested using the 
\verb!^conj($spanverbs, infinitiveform)! function, e.g.


\begin{Verbatim}
^conj($spanverbs, hablar) ;
\end{Verbatim}

You can also test \verb!$spanverbs! in the usual way, invoking


\begin{Verbatim}
test $spanverbs ;
\end{Verbatim}

\noindent
and then entering forms like \emph{canto}, \emph{cantas} and \emph{cantamos}
in the lower-side field of the test window.


\subsection{Spanish Nouns}

This section is currently empty.

\section{Latin Morphology}

This section is currently empty.

\section{Aymara Morphology}

This section is currently empty.
