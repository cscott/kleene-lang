\chapter{\acro{gui} \fsm{} Testing and \acro{other}}

\label{app:test}

In the Kleene \acro{gui}, using the \texttt{test} window, you can type in
individual strings for generation or analysis, and see the output string
or strings displayed in the terminal window.  When testing \fsm{}s
containing \acro{other} (i.e.\@ unknown) labels, the output may at first
be unintuitive. 

The first step is to understand that the \texttt{test} function implements
the formal algorithm for applying an \fsm{} to an input string.  For
generation using \verb!$testFst!

\begin{enumerate}
\item
The input string is built into a one-path \fsm{}; think of it as \verb!$input!.
\item
The input \fsm{} is composed ``on top of'' the \verb!$testFst!, i.e.\@ 
\verb!($input _o_ $testFst!)
\item
Then the output side (also known as the lower side) of the composition is extracted, i.e.\@
\verb!$^lowerside($input _o_ $testFst)!.  The result is an acceptor. 
\item
Then, if the result acceptor is not empty, and the number of paths is finite, the strings are listed in the
terminal window.
\end{enumerate}

\noindent
Similarly for analysis, the result acceptor is computed as 
\verb!$^upperside($testFst _o_ $input)!.

The second important point is that if the \fsm{} being tested contains
\acro{other\_nonid}, then any output strings containing unknown characters
will display as \acro{other\_id} rather than \acro{other\_nonid}.  This is
because when a projection (an acceptor) is extracted, the
\acro{other\_nonid} label cannot appear in an acceptor and is
automatically converted to \acro{other\_id}.

\begin{Verbatim}[fontsize=\small]
$testFst = .:. ;
draw $testFst ;
$generationresult = $^lowerside(z _o_ $testFst) ;
draw $result ;

test $testFst ;	// and enter z on top for generation
\end{Verbatim}

\noindent
In this example, the \verb!$testFst! \fsm{} will contain both
\acro{other\_id} and \acro{other\_nonid} labels, but the
\verb!$generationresult! \fsm{}, and the printed output of testing
\verb!$testFst! using the \texttt{test} facility, and generating
\texttt{z}, will contain \acro{other\_id} and not \acro{other\_nonid}.
Again, the reason is that the application algorithm extracts a projection
of the composition, and a projection is an acceptor and cannot contain
\acro{other\_nonid}.

The final important point in interpreting the output is to understand
that the meaning/coverage of \acro{other\_id} in the result should be
interpreted relative to the \verb!$generationresult! or
\verb!$analysisresult! \fsm{}, which is automatically created and
displayed when using \texttt{test}.

Other kinds of testing, using external runtime code, cannot reasonably
implement the formal composition-plus-projection-extraction method of
application, and the results will be different.  In the future, the
Kleene \texttt{test} facility may offer a choice between the current
formal method and alternative methods used in runtime code.

