\chapter{Pre-defined Case Functions}


The following functions change or augment networks to handle case
variants.  The \verb!$proj! argument must have the value \texttt{input}
(or \texttt{upper}), \texttt{output} (or \texttt{lower}), or
\texttt{both}, which is the default, e.g.

\begin{Verbatim}[fontsize=\small]
$foo = (abc):(def) ;
$bar = $^uc($foo, output) ; // or $^uc($foo, "output")
\end{Verbatim}

\noindent
sets \verb!$bar! to a version of \verb!$foo! that has ``DEF'' rather than
``def'' on the output/lower side.

\vspace{0.5cm}

\noindent
\begin{tabular}{|l|l|}
\hline
\verb/$^uc($fst,/\verb!$proj="both"!\verb!)! & convert to uppercase\\
\verb/$^uc!($fst,/\verb!$proj="both"!\verb!)! & convert to uppercase\\
\verb/$^lc($fst,/\verb!$proj="both"!\verb!)! & convert to lowercase\\
\verb/$^lc!($fst,/\verb!$proj="both"!\verb!)! & convert to lowercase\\
\hline
\verb/$^init_uc($fst,/\verb!$proj="both"!\verb!)! & convert initial char to uppercase\\
\verb/$^init_uc!($fst,/\verb!$proj="both"!\verb!)! & convert initial char to uppercase\\
\verb/$^init_lc($fst,/\verb!$proj="both"!\verb!)! & convert initial char to lowercase\\
\verb/$^init_lc!($fst,/\verb!$proj="both"!\verb!)! & convert initial char to lowercase\\
\hline
\verb/$^opt_uc($fst,/\verb!$proj="both"!\verb!)! & allow/add uppercase\\
\verb/$^opt_uc!($fst,/\verb!$proj="both"!\verb!)! & allow/add uppercase\\
\verb/$^opt_lc($fst,/\verb!$proj="both"!\verb!)! & allow/add lowercase\\
\verb/$^opt_lc!($fst,/\verb!$proj="both"!\verb!)! & allow/add lowercase\\
\hline
\verb/$^opt_init_uc($fst,/\verb!$proj="both"!\verb!)! & allow/add initial uppercase\\
\verb/$^opt_init_uc!($fst,/\verb!$proj="both"!\verb!)! & allow/add initial uppercase\\
\verb/$^opt_init_lc($fst,/\verb!$proj="both"!\verb!)! & allow/add initial lowercase\\
\verb/$^opt_init_lc!($fst,/\verb!$proj="both"!\verb!)! & allow/add initial lowercase\\
\hline
\verb/$^opt_ci($fst,/\verb!$proj="both"!\verb!)! & allow/add uppercase \& lowercase\\
\verb/$^opt_ci!($fst,/\verb!$proj="both"!\verb!)! & allow/add uppercase \& lowercase\\
\verb/$^opt_init_ci($fst,/\verb!$proj="both"!\verb!)! & allow/add initial uppercase \& lowercase\\
\verb/$^opt_init_ci!($fst,/\verb!$proj="both"!\verb!)! & allow/add initial uppercase \& lowercase\\
\hline
\end{tabular}

\vspace{0.5cm}

The \texttt{ci} in the final four names stands for ``case insensitive''
(i.e.\@ accept uppercase or lowercase), and the following functions with
abbreviated names are also provided for convenience.

\vspace{0.5cm}

\noindent
\begin{tabular}{|l|l|}
\hline
\verb/$^ci($fst,/\verb!$proj="both"!\verb!)! & allow/add uppercase \& lowercase\\
\verb/$^ci!($fst,/\verb!$proj="both"!\verb!)! & allow/add uppercase \& lowercase\\
\verb/$^init_ci($fst,/\verb!$proj="both"!\verb!)! & allow/add initial uppercase \& lowercase\\
\verb/$^init_ci!($fst,/\verb!$proj="both"!\verb!)! & allow/add initial uppercase \& lowercase\\
\hline
\end{tabular}

\vspace{0.5cm}

It is anticipated that the behavior of the ``ci'' will be re-examined and
perhaps changed, and that new functions will be added in future releases
to handle ``title''-casing and ``camel''-casing.

