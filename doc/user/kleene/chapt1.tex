\chapter{Introduction}

\section{What is Kleene?}

\Kleene{} is a programming language for building and manipulating
finite-state machines (\fsm{}s), and these finite-state machines can be
used as tokenizers, spelling checkers, spelling correctors,
morphological analyzer/generators, shallow parsers, speech generators
and speech recognizers.
Applications implemented using finite-state machines are mathematically
beautiful, unusually modifiable and re-usable, and typically more
efficient than the alternatives.  This book will concentrate mostly on
showing you how to define
\fsm{}s that implement morphological analyzer/generators.

The advantages of computing with finite-state machines are well known,
and numerous computer implementations have been developed.  Dr.\@ Anssi
Yli-Jyrä maintains a formidable list of finite-state projects at
\url{https://kitwiki.csc.fi/twiki/bin/view/KitWiki/FsmReg}.  Kleene is
firmly in the tradition of the \init{at\&t} Lextools
\citep{roark+sproat:2007},\footnote{\url{http://serrano.ai.uiuc.edu/catms/}}
the \init{sfst-pl} language
\citep{schmid:2005},\footnote{\url{http://www.ims.uni-stuttgart.de/projekte/gramotron/SOFTWARE/SFST.html}}
The \init{hfst}
software,\footnote{\url{http://www.ling.helsinki.fi/kieliteknologia/tutkimus/hfst/}}
the Xerox/\acro{parc} finite-state toolkit
\citep{beesley+karttunen:2003};\footnote{\url{http://www.fsmbook.com}}
the \acro{foma} language\footnote{https://code.google.com/p/foma/}
and
OpenGrm-Thrax,\footnote{http://openfst.cs.nyu.edu/twiki/bin/view/GRM/Thrax}
all of which provide higher-level programming formalisms built on top of
low-level finite-state libraries.  

\Kleene{} allows programmers to
specify finite-state machines, including acceptors that encode
regular languages and two-projection transducers that encode regular
relations, using both regular expressions and right-linear
phrase-structure grammars. The \fsm{}s can be weighted, under the
Tropical Semiring, or unweighted.  The language supports variables, 
user-defined functions and a rich set of alternation-rule notations, 
plus familiar control structures such as \verb!if-elsif-else!
statements and \verb!while! loops.\footnote{Kleene was first reported
at the 2008 conference on Finite-State Methods in Natural-language
Processing \citep{beesley:2009}.}

The Java-language \Kleene{} parser, implemented with JavaCC and
JJTree \citep{copeland:2007},\footnote{\url{https://javacc.dev.java.net}} 
is Unicode-capable and portable.\footnote{Successfully parsed Kleene statements are reduced to abstract
syntax trees (\init{ast}s); and the Java-language interpreter,
implemented using the visitor design
pattern,\footnote{\url{http://en.wikipedia.org/wiki/Visitor_pattern}} 
interprets the \init{ast}s by calling \CPP{} functions in the
OpenFst finite-state 
library
\citep{allauzen+riley+schalkwyk+skut+mohri:2007}
via the Java Native Interface
(\init{jni}) \citep{gordon:1998,liang:1999}. } Kleene currently has
pre-compiled binaries available for
OS X and Linux.

The \Kleene{} graphical user interface (\acro{gui}), implemented with 
the Java Swing library, 
allows interactive creation, testing and graphic display
of finite-state machines.  The overall application window encloses a Unicode-capable terminal-like window, into which Kleene statements
can be typed,
and a symbol-table window that displays an icon for each defined
machine.  Right-clicking on an icon triggers a pop-up menu with
commands to test, draw, invert, determinize, draw, delete, 
etc.\@ the associated machine.

\Kleene{} is copyrighted by \init{sap} \init{AG} and released under the Apache License, Version 2, which
is a very liberal license that allows even commercial usage without payment of license fees or
royalties.  The OpenFst Library on which Kleene is based is also released under the same Apache License,
Version 2.  See the Apache site at \url{http://www.apache.org/licenses/} for the full details.

\section{The Kleene Name}

\Kleene{} is named in honor of American mathematician Stephen Cole Kleene
(1909--1994), who, among many accomplishments, investigated the properties of regular sets, including
regular languages, and invented the metalanguage of regular expressions, which is the foundation of
\Kleene{}-language syntax.  In the \init{usa}, the name Kleene is
commonly pronounced /kliːn/, like the English word \emph{clean}, 
or /ˈkliːni/, but Kleene himself reportedly pronounced it
/ˈkleɪni/.\footnote{\url{http://en.wikipedia.org/wiki/Stephen_Cole_Kleene}}

\section{Possible Applications}

Kleene can build acceptors to be used in spell-checking and similar
applications.  Kleene transducers can be used in spelling correction,
tokenization, phonological modeling, morphological analysis and
generation, speech synthesis and generation, and shallow or ``robust''
parsing.  [KRB:  expand this section]

\section{Design Criteria}

The following requirements and desiderata have guided the design and
implementation of \Kleene{}.  You will have to judge if the project has
succeeded.

\begin{enumerate}

\item
The \Kleene{} language must be compelling, easy to learn and well documented.  
The syntax and semantics should always aim for maximum
familiarity and ``least astonishment.''

\item
Programmers must be able to run pre-edited scripts or
type statements interactively into a \acro{gui}.

\item
\Kleene{} must allow finite-state machines to be defined using 
both regular expressions and right-linear
phrase-structure grammars. 

\item
The syntax should follow, as far as
possible and appropriate, the familiar syntax of Perl-like regular
expressions.\footnote{This was a fundamental criterion from the
very beginning, and Mike Wilkens correctly identified it
as the major reason that Kleene looks the way it does.}  Non-regular features of Perl regular expressions, 
such as back-references, will be
excluded; and operators will be added to denote weights,
two-projection relations, subtraction, complementation and intersection, which are
lacking in Perl-like regular expressions.

\item
The abstraction mechanisms must include variables, built-in functions
and user-defined functions.


\item
The syntax will include rule-like expressions similar in semantics and 
function to the
Xerox/\acro{parc} Replace Rules
\citep{karttunen:1995,karttunen+kempe:1995,karttunen:1996,kempe+karttunen:1996,mohri+sproat:1996} that denote regular relations and
compile into finite-state transducers.  

\item
Unicode must be supported from the beginning, not only in data
strings, but also in \Kleene{} identifiers and operators.

\item
The implementation should be maximally portable.

\item
The implementation should be maximally modular, allowing the writing of
interpreters based on various finite-state libraries that might become
available.
\end{enumerate}

\section{Terminology}

A \emph{regular language} is a set of strings that can be encoded as a
finite-state acceptor.  A \emph{regular relation}
is a set of ordered string pairs that can be
encoded as a two-projection (or ``two-tape'') finite-state transducer.
Theoretically, transducers can have any number of projections, but most
practical implementations---including the
Xerox/\acro{parc}, \init{at\&t}, \init{sfst} and OpenFst implementations---have been limited to two.  Kleene can be used to build finite-state acceptors
and two-projection finite-state transducers.

In some traditions, the terms \emph{finite-state machine} and \emph{finite-state automaton}
(plural: \emph{automata}) are used to
denote acceptors, in contrast to finite-state \emph{transducers}.  Herein, however, I avoid the
term automaton and use the term \emph{finite-state machine} to encompass both
acceptors and transducers.\footnote{In the Xerox/\acro{parc} tradition, the term
\emph{network} is used as a cover term to include both acceptors and transducers, but this
usage has not become general.}

I dislike acronyms and initialisms, especially as they have proliferated and become
increasingly
ambiguous; but some are well established and almost unavoidable in the field.  I will often use
\emph{\fst{}} to refer to finite-state transducers, and \emph{\fsm{}} to refer to finite-state machines
(encompassing both finite-state acceptors and transducers).
Where it is important to distinguish a finite-state machine as an acceptor, it will be
called an acceptor, without abbreviation.

Complicating the terminological issue is the fact that OpenFst finite-state machines are always two-projection
transducers in their structure; each arc in an OpenFst finite-state machine has an ``input
label'' and an associated ``output label.''  In the OpenFst
implementation of finite-state machines, an acceptor is just a special
case of a transducer, which can also be considered an \emph{identity transducer}, wherein each input label is equal to
its associated output label.  These structural issues, and the terminology, will be
discussed again in more detail below in a section on how finite-state machines are
implemented and visualized in various traditions.


