
\chapter{Unicode Support}

\section{Kleene, Java and Unicode}

The Kleene parser is a Java program, and Kleene supports Unicode to the extent (and
in the same way) that any Java program supports Unicode, which is pretty well---but
not perfectly.\footnote{The Unicode Standard (\url{http://www.unicode.org})
is mature and very comprehensive, but
the \emph{implementation} of Unicode in programming languages, 
text editors, \acro{gui} libraries,
typesetting packages, and other text-handling software varies considerably in completeness
and reliability.}  The
Kleene \acro{gui} is written using the Java Swing library, and Swing text
widgets---including JTextField and JTextArea---are automatically Unicode-friendly.  

\section{Kleene Scripts and Unicode}

\subsection{The Default Encoding of the Operating System}

Pre-edited Kleene scripts can be run from the command line, or from the \acro{gui}.
It is \emph{not} required that Kleene scripts be stored as Unicode---Kleene can
read and execute scripts written in a huge number of standard encodings, converting the text to Unicode for
internal processing.  If Kleene is told to run a script, and if the encoding of that script is not
explicitly specified, then by default it will assume that the script is in the
default encoding of the operating system, whatever that might be.

\begin{Verbatim}
$ java -jar Kleene.jar myscript
\end{Verbatim}

\noindent
In this example, \texttt{myscript} would be assumed to be stored in the default encoding
of the operating system and would be opened and read as such, with Java performing
an automatic conversion of the text from that default encoding to Java's internal
Unicode encoding, which happens to be \acro{utf}-16.

Note that Kleene does not attempt to analyze the input file and detect what its encoding
might be from internal evidence.  Rather it interrogates the operating system to
find the default encoding, and it attempts to open and read the file accordingly.  If, for some reason,
the input file is not in the default encoding of the operating system, the input
may result in a warning message or garbling.

In a Unix-like system, the default encoding of the operating system can be seen by
entering \texttt{locale} at the command line:

\begin{Verbatim}
$ locale
\end{Verbatim}

\noindent
On any system, the default encoding, as seen by a Java program, can also be revealed by compiling
and running the following trivial Java script, which should be in a file named
\texttt{FindDefaultEncoding.java}.

\begin{Verbatim}
public class FindDefaultEncoding {
    public static void main(String[] args) {
        String s = System.getProperty("file.encoding") ;
        System.out.println(s) ;
    }
}
\end{Verbatim}

\noindent
To compile and run this script, do the following:

\begin{Verbatim}
$ javac FindDefaultEncoding.java
$ java FindDefaultEncoding
\end{Verbatim}

On \acro{os x} even if the locale has been set to some specified
encoding, such as \acro{utf-8}, the call to
\verb!System.getProperty("file.encoding")! will still, unfortunately,
return MacRoman.\footnote{Curiously, in the Groovy and Scala
languages, which are, like Java, based on the Java Virtual Machine
(\acro{jvm}), this problem appears to have been fixed.  On a
system where the locale is set to \acro{utf-8}, executing
\texttt{println (System.getProperty("file.encoding"))} in Scala
2.8.1 or Groovy 1.7.5 returns \verb!"UTF-8"!.}  This can be overcome by calling any Java program with a
\verb!-D! option specifying the desired default
encoding.

\begin{Verbatim}
$ javac FindDefaultEncoding.java
$ java -Dfile.encoding=UTF-8 FindDefaultEncoding
\end{Verbatim}

\subsection{Running Scripts in Any Standard Encoding}

Regardless of the default encoding of the operating system, Kleene can
run scripts in any standard encoding as long as that encoding is
explicitly specified using the \texttt{-encoding} flag.\footnote{This
\texttt{-encoding} flag, and its semantics, are copied from the same flag
used for the \texttt{javac} compiler; see
\url{http://java.sun.com/j2se/1.5.0/docs/tooldocs/windows/javac.html}.}
So if the script is, for some reason, stored in an encoding that is not
the default encoding of the operating system, it is necessary to specify
that encoding as in the following examples: 

\begin{Verbatim}
$ java -jar Kleene.jar -encoding UTF-8      myscript
$ java -jar Kleene.jar -encoding UTF-16     myscript
$ java -jar Kleene.jar -encoding Latin-1    myscript
$ java -jar Kleene.jar -encoding ISO-8859-1 myscript
$ java -jar Kleene.jar -encoding ISO-8859-6 myscript
$ java -jar Kleene.jar -encoding EUC-JP     myscript
\end{Verbatim}

\subsection{A Plug for Unicode}

All things being equal, users working on languages with orthographies
that cannot be represented in \acro{ascii} are highly encouraged to use
Unicode rather than resorting to obsolete 8-bit encodings or, especially,
to Roman transliterations that may have been required in pre-Unicode
software.\footnote{The \acro{e-meld} School of Best Practices recommends
the use of Unicode for all textual archiving:
\url{http://emeld.org/school/bpnutshell.html};
\url{http://emeld.org/school/classroom/unicode/index.html}.}

\section{Typing Unicode Characters into Kleene \acro{gui} Text Widgets}

\subsection{Unicode-capable Text Widgets}

Because the Kleene \acro{gui} is written in Java/Swing, the text widgets
are automatically Unicode-capable and sensitive to standard Java Input
Methods.  At any time when typing text into a Swing text widget, you can
select (or ``activate'') a particular Java Input Method of your choice
(as long as it's installed in your Java environment) to facilitate typing
in Unicode characters for 1) European scripts with various accents, 2)
\acro{ipa}, 2) Greek, 3) Russian, 4) Arabic, 5) Chinese or whatever.  You
can switch from one input method to another at any time.  In their
simplest form, Java Input Methods define a straightforward remapping of
the keyboard; they can also support code-point (hex) input of characters,
dead-key sequences, transliteration-based input methods, and more
challenging dialog-based input methods for the Chinese/Japanese/Korean
scripts.

\subsection{Java Input Methods}

Java Input Methods are standard, well-documented, cross-platform, and
often freely available; some useful Java Input Methods will be
distributed with Kleene.  You can install and use your own favorite Java
Input Methods in your own Java installation.  The key documentation on
installing and selecting Java Input Methods, for users, is ``Using Input
Methods on the Java Platform'', by Naoto
Sato.\footnote{\url{http://javadesktop.org/articles/InputMethod/index.html}}



\subsection{Installing Java Input Methods}

To run any Java program, including Kleene, you need to have a Java
installation on your system.  If you can run Kleene at all, you have such
a Java installation.  To see which version of Java you have installed,
enter

\begin{Verbatim}
$ java -version
\end{Verbatim}

\noindent
You should be running Java 1.5 or higher.

The root of your Java installation should (on most platforms) be pointed
at by the environment variable \acro{java\_home}.  

\begin{Verbatim}
$ echo $JAVA_HOME
\end{Verbatim}

\noindent
On \acro{os~x}, \texttt{/Library/Java/Home} should be a link to the root
of the Java installation, within the rather complicated hierarchy of
\acro{os~x} frameworks.  So on \acro{os~x} if you need to define
\acro{java\_home}, set it to \texttt{/Library/Java/Home}.

Your Java installation has an Extension Directory where you can install
Extensions, including Java Input Methods.\footnote{If you work on a
network, with a shared Java installation, you might not have permission
to copy extensions to the extension directory, and then you would need to
contact your administrator for help.}  On \acro{os~x}, user-installed
Java extensions are most safely put in
\texttt{/Library/Java/Extensions/}; this directory stays stable when you
upgrade to new versions of Java.  On any system, compile and run the
following Java program to find the extension directory or
directories.

\begin{Verbatim}
public class FindExtensionDirectory {
   public static void main(String[] args) {
       System.out.println(System.getProperty("java.ext.dirs")) ;
   }
}
\end{Verbatim}
 
\noindent
To compile and run this script, which should be stored in a file
named \texttt{FindExtensionDirectory.java}:

\begin{Verbatim}
$ javac FindExtensionDirectory.java
$ java FindExtensionDirectory
\end{Verbatim}

\noindent
The output shows the directory, or set of directories, where you can
install new extensions.\footnote{Running this program on my Linux system
produced the output 
\texttt{/usr/java/jdk1.6.0\_16/jre/lib/ext/usr/java/packages/lib/ext}, 
showing two extension directories.  In this
case, putting Java Input Methods in \texttt{/usr/java/packages/lib/ext}
is safer because they will be unaffected by future Java updates.} Once an
input method is installed in the extension directory of the \acro{jvm},
it is automatically visible to all Java programs running that \acro{jvm},
without recompilation, without resetting your \acro{classpath}, and
without use of the \verb!-D! command-line flag.  

Java usually comes complete with some built-in input methods, such as
\texttt{CodePointIM.jar}, which is found in
\url{\$JAVA_HOME/demo/plugin/jfc/CodePointIM/} or
\url{\$JAVA_HOME/demo/jfc/CodePointIM/}.  This should make \texttt{CodePointIM.jar}
available without you having to install anything. 

Some pure Java Input Methods are readily downloadable:

\vspace{.5cm}

\renewcommand\tabcolsep{1.25mm}

\noindent
\begin{tabular}{|l|l|}
\hline
CodePointIM.jar     & enter Unicode characters by Unicode code-point value\footnotemark \\
\hline
zh\_pinyin.jar      & Chinese\footnotemark \\
\hline
vietIM.jar          & Vietnamese\footnotemark \\
\hline
BrahmaKannadaIM.jar & Brahmi\footnotemark \\
\hline
BibleIM.jar         & Hebrew and Greek\footnotemark \\
\hline
\end{tabular}

\vspace{.5cm}

\addtocounter{footnote}{-4}
\footnotetext{\url{http://www.grogy.com/local_doc/www/apache22/data/local_doc/jdk1.6.0/demo/jfc/CodePointIM/},
\url{http://courses.cs.tau.ac.il/databases/workshop/jdk1.6.0_01/jdk1.6.0_01/demo/jfc/CodePointIM/}}

\stepcounter{footnote}
\footnotetext{\url{http://www.chinesecomputing.com/programming/java.html}}

\stepcounter{footnote}
\footnotetext{\url{http://vietunicode.sourceforge.net/howto/unicodevietime.html},
\url{http://sourceforge.net/projects/vietime/}, \url{http://vietime.sourceforge.net/usage.html}}

\stepcounter{footnote}
\footnotetext{\url{http://sourceforge.net/projects/brahmi}}

\stepcounter{footnote}
\footnotetext{\url{http://code.google.com/p/bibleunicodepad/}}


Reasonably skilled Java programmers can even write their own custom Java
Input Methods, but it's not painless.\footnote{A Java Input Method is a
Java class that implements the InputMethod interface.  For information on
writing new input methods, and packaging them properly as extensions, see
the book \emph{Java Internationalization}
\citep{deitsch+czarnecki:2001}.} Luckily, as an easy alternative, there is a very useful
meta-Java Input Method, installable as \texttt{kmap\_ime.jar} and
\texttt{kmap\_ime\_gui.jar}, that allows you to use simple Yudit
\texttt{kmap} files\footnote{They also allow you to use Simredo
\texttt{kmp} files, which I haven't used myself.} as if they were Java
Input Methods.  Yudit \texttt{kmap} files are much easier to write than
pure Java Input methods, and many are already included with
\texttt{kmap\_ime}.  These kmap meta-Java Input Methods are treated in
detail below.


\subsection{The CodePointIM.jar}

When you need to enter a single exotic character, or even just a word or
two, it is often easiest to simply enter each character by its code point
value.  When the \texttt{CodePointIM.jar} is selected, you can continue
to type \acro{ascii}-range characters as normal, but the sequence 

\begin{Verbatim}
\uHHHH
\end{Verbatim}

\noindent
where
\texttt{HHHH} is exactly four hexadecimal digits, is intercepted,
and the single Unicode character with the code point
value \texttt{HHHH} is entered into the buffer.  Similarly, the sequence 

\begin{Verbatim}
\UHHHHHH
\end{Verbatim}

\noindent
with an uppercase \texttt{U} followed by exactly six hexadecimal digits,
is intercepted and the single supplementary Unicode character with the
code point value \texttt{HHHHHH} is entered into the buffer.

Note that Kleene, like Python, accepts the \verb!\UHHHHHHHH!, notation,
which requires \emph{eight} hex digits rather than the \emph{six}
required by the CodePointIM.  Kleene also accepts the
\verb!\U{H!\ldots\verb!}! notation, which allows one or more hex digits
between the curly braces.)

\subsection{The \texttt{kmap} Meta-Java Input Method}

There's a very useful meta-Java Input Method called \texttt{kmap\_ime},
or \texttt{kmap} for short; the extension files \texttt{kmap\_ime.jar}
and \texttt{kmap\_ime\_gui.jar} can be downloaded from
\url{http://sourceforge.net/projects/jgim}.  Install these extensions in
a Java extension directory as usual.  The \texttt{kmap} extension allows
you to use Yudit kmap input methods as if they were pure Java Input
Methods.  It is much easier to write a Yudit kmap file than it is to
write a new Java Input Method.  Many kmap files are freely available, and
instructions for writing new Yudit kmap files can be found at
\url{http://www.yudit.org/en/howto/keymap/}.

To be visible to this meta-Java Input Method, your Yudit kmap files
should be placed in the directory \texttt{\~{}/kmap/}.  The
\texttt{kmap} download
comes complete with a \texttt{kmap/} directory containing over 200 kmap
files, defining input methods for everything from Albanian to Yoruba.  Of
course, any single user will probably need only a small subset of those
input methods, and perhaps a different subset from time to time.  It also
appears that the input-method-selection dialog does not allow scrolling
through the 200 possibilities.  One reasonable approach is to keep the
full set of input methods available in a repository directory, named
something like \texttt{\~{}/kmap.all/}, and copy a desired subset of the
kmap files  to \texttt{\~{}/kmap/}.  I.e., starting with the
\texttt{kmap/} directory as downloaded:

\begin{Verbatim}
$ cp -r kmap ~/kmap.all
$ cd ~
$ mkdir kmap
\end{Verbatim}

\noindent
and then copy some desired subset of the kmap files from
\texttt{\~{}/kmap.all/} to \texttt{\~{}/kmap/}.  Put any new kmap
files that you write or modify yourself into \texttt{\~{}/kmap/}.  
Again, only the kmap
files in \texttt{\~{}/kmap/} will be visible to the \texttt{kmap} meta-Java Input
Method.

The \texttt{kmap} download also includes a \texttt{simredo/} directory containing
over 100 input methods in the Simredo kmp format.  \texttt{Kmap} can recognize and
use these kmp files as well.  To make these Simredo files visible to
\texttt{kmap}, create a \texttt{\~{}/simredo/} directory on the model of the
\texttt{\~{}/kmap/} directory just described.

\section{Selecting a Java Input Method}

How you select a Java Input Method depends on your platform, and this
subject is treated in detail in the Sato paper cited above.  When running
Java on Solaris and Windows, the \texttt{System} pull-down menu
automatically contains a menu item allowing you to select one of the
installed Java Input Methods.

On Linux and \acro{os~x} it's not quite so easy; 
you need to define a ``hot key'' that triggers a
pop-up menu that allows you to select one of the installed Java Input
Methods.  The \acro{jar} file \texttt{InputMethodHotKey.jar}, used
to set the hot key, can be
downloaded from the Sato paper cited, and it is run as

\begin{Verbatim}
$ java -jar InputMethodHotKey.jar
\end{Verbatim}

\noindent
You can re-run \texttt{InputMethodHotKey.jar} as often as desired
or necessary to reset the hot key.
Again, the setting of the hot key is necessary only on Linux and
\acro{os~x}.\footnote{For example, I use Ctrl-i as my hot key on Linux
and \acro{os x}.  When \texttt{InputMethodHotKey.jar} is launched, a
little dialog box is displayed showing the current setting, if any.  To
set or reset the hot key, wait for the box to be displayed and then press
Ctrl-i, or whatever, directly on your keyboard. Then close the dialog box
and restart Java.}

When you select the \texttt{kmap} input method (via the \texttt{System}
menu, or via a hot key), the menu lets you make a secondary selection of
one of the Yudit kmap files installed in your \texttt{\~{}/kmap/}
directory, or one of the Simredo kmp files installed in your
\texttt{\~{}/simredo/} directory.

\section{Rendering of Unicode Characters in the \acro{gui}}

Entering Unicode characters into a Swing text widget is one
thing---seeing them \emph{rendered} properly on the screen is another.
Whether your Unicode characters render properly will depend on the fonts
installed in your Java installation, and on the sophistication of the
rendering engines in the Swing text widgets themselves.

Font installation varies by platform and by Java version; see your
documentation.  Rendering engines are an area where
implementations, including the allegedly Unicode-savvy Swing text
widgets, can be disappointing.  The more ``exotic'' the script,
the less likely the text widgets will render them properly.  If
rendering becomes an issue with your applications, edit 
Kleene source files outside the \acro{gui} using your favorite Unicode-savvy text editor,
and run them as scripts.

\begin{quote}
\textbf{KRB Note to myself}: Try to be more helpful here.  In my own
experience with Swing Text Widgets, the rendering of letters with
Combining Diacritical Marks is disappointing, sometimes unacceptable.  Of
course, Swing Text Widgets \emph{should} render Combining Diacritical
Marks acceptably, so it is up to me and others to lobby for improvements.
\end{quote}

\section{Java Input Method Conclusions}

Preferences about input methods, as about text editors, are intensely
personal and often almost religious.  My own needs for entering Unicode
include the International Phonetic Alphabet (\acro{ipa}), Roman letters
with unusual Combining Diacritical Marks, and very occasionally Greek,
Cyrillic and Georgian letters.  Some non-\acro{ascii} mathematical
symbols can be used in Kleene syntax itself.  And I insist on being able
to write my own input methods to match my needs and taste.

With this background, my favorite approach to entering exotic Unicode
characters in Kleene, and in any Java \acro{gui}, is the use of the
\texttt{kmap\_ime.jar} and \texttt{kmap\_ime\_gui.jar} meta-input
methods, along with easily defined and edited Yudit kmap files.  I will
eventually deliver with Kleene a kmap file that facilitates typing in
some of the Unicode characters used in Kleene syntax, such as the epsilon
$\epsilon$ and the $\circ$ symbol that can be used to indicate
composition.  You can also type in the \acro{ascii} \verb!_e_! and the
\verb!_o_! for these (and Kleene will parse the \acro{ascii} forms
without trouble); but the Kleene kmap (when active) will simply intercept
the typed sequences \verb!_e_! and \verb!_o_! and turn them into the real
epsilon and composition-operator characters, respectively.
 

