\chapter{Optimization of Networks}

\label{app:optimize}

\section{Automatic Optimization}

The default behavior of Kleene is to optimize each network by calling
the OpenFst Determinize(), Minimize() and RmEpsilon() functions wherever
these operations are mathematically safe.  Note that an operation that is
mathematically safe can still overwhelm the available memory or
take a very long time to complete.

In the Kleene interpreter, the Java function \texttt{optimizeInPlace()}
is automatically called to optimize each new network, and it in turn
calls a native \CPP{} function called \texttt{optimizeInPlaceNative()},
which calls the OpenFst functions 
Determinize(), Minimize() and RmEpsilon(), where
mathematically safe.

\section{Suppressing Automatic Optimization}

In some cases it may be desirable or necessary to suppress all or part of
the automatic optimization that Kleene performs by default.  For example,
developers and advanced users might want to see what a network looks like
before and after optimization.  More critically, in real applications the
networks can easily get very large, and it may be necessary to suppress
some optimization, especially the determinization step, to keep the
networks from blowing up in size and/or taking forever to finish.

Kleene defines (in the system-supplied 
start-up file \verb!~/.kleene/global/basic.kl!) the following three
special numeric variables which are interpreted as booleans (true or
false):

\begin{itemize}
\item
\#KLEENEdeterminize
\item
\#KLEENEminimize
\item
\#KLEENErmepsilon
\end{itemize}

\noindent
These variables display a naming convention that puts \textbf{KLEENE} in front
of system variables that control the behavior of Kleene itself.

These variables can be set and interrogated in the usual ways, e.g.

\begin{Verbatim}[fontsize=\small]
#KLEENEdeterminize = 0 ;	// to turn off just determinization
#KLEENEdeterminize = 1 ;
or equivalently
#KLEENEdeterminize = #false ;  // given that #false is set to 0
#KLEENEdeterminize = #true ;

if (#KLEENEdeterminize) {
    // then do something
} else {
    // do something else
}
\end{Verbatim}

\noindent 
These variables affect the behavior of the \texttt{optimizeInPlace()}
algorithm, which is called routinely on newly created networks.

Also defined in basic.kl are the following convenience functions:

\begin{Verbatim}[fontsize=\small]
^setOptimize(#bool)
	which sets #KLEENEdeterminize, #KLEENEminimize and
	#KLEENErmepsilon to the passed in #bool value, 
	
e.g.
^setOptimize(#false) ;
\end{Verbatim}

\noindent
also

\begin{Verbatim}[fontsize=\small]
^setDeterminize(#bool) ;
^setMinimize(#bool) ;
^setRmepsilon(#bool) ;
\end{Verbatim}

The global variables \verb!#KLEENEdeterminize!, \verb!#KLEENEminimize!
and \verb!#KLEENErmepsilon!, defined in the global start-up file
\texttt{basic.kl}, can be shadowed by user-defined variables of the same
name that are local to function code blocks and stand-alone code blocks.
So programmers can play the scope game when turning these variables on
and off.

\section{User-callable Optimization Functions}

Kleene provides built-in net-valued functions that take a network
argument and return a new optimized network as the result.  These
functions are for use when the default optimization has been turned off.
The non-destructive functions are

\begin{Verbatim}[fontsize=\small]
$^optimize(regexp)
$^rmEpsilon(regexp)
$^determinize(regexp)
$^minimize(regexp)
\end{Verbatim}

\begin{Verbatim}[fontsize=\small]
$^synchronize(regexp)
\end{Verbatim}

\noindent
and the destructive versions are

\begin{Verbatim}[fontsize=\small]
$^optimize!(regexp)
$^rmEpsilon!(regexp)
$^determinize!(regexp)
$^minimize!(regexp)
\end{Verbatim}

\begin{Verbatim}[fontsize=\small]
$^synchronize!(regexp)
\end{Verbatim}

\noindent
The \verb!$^optimize()! and \verb/$^optimize!()/ functions cause---where
mathematically safe---determinization, minimization and epsilon-removal to
be invoked.  More specific functions such as \verb!$^determinize()! and
\verb/$^determinize!()/ are also performed only when the operation is
mathematically safe.

Kleene also supplies the following statements, which take one or more
network-ID arguments, optionally separated by commas, and optimize them
in place (destructively).

\begin{alltt}
optimize $a, $b, ... ;
determinize $a, $b, ... ;
minimize $a, $b, ... ;
rmEpsilon $a, $b, ... ;
\end{alltt}

\begin{alltt}
synchronize $a, $b, ... ;
\end{alltt}

Just as the \texttt{rmEpsilon} command and the \verb!$^rmEpsilon()!
function remove two-sided \verb!$eps:$eps! arcs, the \texttt{synchronize}
command and the \verb!$^synchronize()! function return networks with a
minimal number of one-sided epsilon arcs.  For example, if the input
network contains a four-arc path with the four labels \verb!a:$eps!,
\verb!$eps:b!, \verb!c:$eps! and \verb!$eps:d!, in that sequence, the
synchronized result path has only two labels, \verb!a:b! and \verb!c:d!.
Note that the synchronized network is equivalent to the input
network---i.e.\@ it encodes the exact same relation; so synchronization
is a lossless way to minimize the arcs in a network.  As currently
implemented, only acyclic networks can be synchronized. Cyclic
networks can be synchronized only if the cycles (loops) themselves have
equal input and output lengths, i.e.\@ only if the cycles do not contain
one-sided epsilons.  KRB:  I'm not sure yet how to test a cyclic network to
make sure that the loops are consistent with this restriction.


