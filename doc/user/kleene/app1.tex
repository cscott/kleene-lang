\chapter{Alphabet (Sigma) and \acro{other}/\acro{unknown} Characters}

\label{app:other}

Kleene automatically keeps track of the set of symbols that are ``known''
to each network, and this set is known traditionally as the alphabet or \emph{sigma}.
Each Kleene network carries its own private sigma.\footnote{The OpenFst
library does not maintain a private sigma for each network; this
functionality is added at the Java level of Kleene.} In the simple case 

\begin{Verbatim}[fontsize=\small]
$v = abc ;
\end{Verbatim}

\noindent
the sigma of network \$v will contain the symbols \emph{a}, \emph{b},
\emph{c} and no other symbols.  The Kleene programmer never has to
declare sigmas manually.

In Kleene regular expressions, the . (dot) special syntactic symbol
semantically represents \emph{any} symbol, and has a non-trivial
semantics.  Intuitively, the notion of \emph{any} symbol properly
includes all known symbols in the sigma, plus the infinite set of
\acro{other}, also known as \acro{unknown}, symbols.  We will use the
term \acro{other} herein. 

Kleene transducers must also distinguish between the \emph{identity
mapping} of \acro{other} symbols vs.\@ the \emph{non-identity mapping} of
\acro{other} symbols.  The identity mapping maps any \acro{other} symbol
to itself, while the non-identity mapping maps any \acro{other} symbol to
any \acro{other} symbol \emph{except} itself.  The difference is best
seen in concrete examples.


When . (dot) appears in a regular expression like

\begin{Verbatim}[fontsize=\small]
$w = . ;
\end{Verbatim}

\noindent
it is interpreted to produce the arc label
\acro{other\_id}:\acro{other\_id}, which represents the \emph{identity
mapping} of \acro{other} symbols, i.e.\@ the mapping of any \acro{other}
symbol to itself.  As the sigma of ``known'' symbols in \$w is empty, the
resulting network maps \emph{a} to \emph{a}, \emph{b} to \emph{b},
\emph{c} to \emph{c}, and similarly for all possible symbols---but not
\emph{a} to \emph{b} because this would be a non-identity mapping.

To denote the mapping of any symbol to any symbol, including itself, the
Kleene syntax .:. is used in regular expressions, e.g.

\begin{Verbatim}[fontsize=\small]
$y = .:. ;
\end{Verbatim}

\noindent
The resulting network for \$y contains a start state and a final state,
linked by two arcs labeled \acro{other\_id}:\acro{other\_id} and
\acro{other\_nonid}:\acro{other\_nonid}, respectively.  As the name
implies, \acro{other\_nonid}:\acro{other\_nonid} represents the
non-identity mapping of \acro{other} symbols.  The two arcs therefore
handle the cases of identity mapping and non-identity mapping.

In Kleene networks, the symbol coverage of \acro{other\_id} and
\acro{other\_nonid} are identical; it can be thought of as \acro{other},
i.e.\@ the set of all possible symbols not in the sigma of the network;
the labels \acro{other\_id}:\acro{other\_id} and
\acro{other\_nonid}:\acro{other\_nonid} differ only in their mapping
behavior: identity mapping vs.\@ non-identity mapping.

When two networks are combined, via operations like union, concatenation
and composition, the result is a new network with its own sigma.  Where
one or both networks to be combined ``contain \acro{other}'', the
operation, and the calculation of the new sigma, can be quite
complicated, but the programmer never needs to worry about it.

The notion of \emph{any symbol}, denoted . (for ``map any other symbol to
itself'') or .:. (``map any other symbol to any other symbol, including
itself'') are syntactic notions that appear in regular expressions.  The
notions of identity mapping, non-identity mapping and the labels
\acro{other\_id}:\acro{other\_id} and
\acro{other\_nonid}:\acro{other\_nonid} belong to underlying networks.  

The arc labels \acro{other\_id} and \acro{other\_nonid} are special, and
these two special symbols should not appear in regular expressions.


