\chapter{Challenges in Morphology}

\section{Overview}

As we have seen, building \fsm{}s for
natural language morphology requires modeling both
the morphotactics and the phonological and/or orthographical alternations.
In morphotactics, the most common word-building processes are prefixation and suffixation,
which translate straightforwardly into the finite-state operation of concatenation and seldom
cause much trouble in Kleene and similar finite-state frameworks.  

Of course, as soon as concatenative morphotactics was shown to be tractable, even for
richly agglutinative languages such as Finnish, Basque, Turkish and Aymara, academic attention
quickly turned to a range of more challenging phenomena, including \emph{infixation}, \emph{reduplication} and Semitic
\emph{interdigitation}.  Another general morphotactic challenge, even in concatenative languages, is the 
constraint of separated (``long-distance'') dependencies within a word, e.g.\@ when a particular prefix
requires a particular suffix, or when a particular prefix is incompatible with a particular suffix.  This
chapter, which is definitely work in progress, will address such challenges in modeling
natural-language morphology in Kleene.

\section{Infixation}

This section is currently empty.

\section{Reduplication}

\subsection{Range of Reduplicative Phenomena}

Whereas prefixation, suffixation and infixation involve concatenating or inserting 
some relatively static morpheme onto/into
a root or stem, reduplication is a morphotactic process in which the root or stem, or part of it,
is \emph{copied} into another part of the resulting word.  The reduplicated parts, sometimes called
\emph{reduplicants}, may be prosodic feet, syllables,
fairly arbitrary sequences of consonants and vowels, or single consonants or vowels, depending on
the language.
It is even possible to have \emph{triplication}, where two copies are made in the same word.

It is beyond the scope of this book to go into all the semantics of reduplication.  Very commonly, a
reduplicated noun marks it as plural or genuine/authentic; a reduplicated verb can be emphatic or convey
some aspectual meaning labeled imperfective, repetitive, frequentative or habitual.

Kleene currently 
offers two predefined functions, 
both based on algorithms kindly 
provided by Dr.\@ M\r{a}ns Huldén,\footnote{Personal communications and 
\citet{Hulden:2009:SFC:1564035.1564059}.} for modeling reduplication:  \verb!$^redup($lang, $sep="")!, which
serves to model the majority of reduplicative phenomena, and \verb!$^eq($fst, $left, $right)!, which
can be used to model more difficult cases.  \verb!$^redup()! in fact
calls \verb!$^eq()!, and other convenience functions could easily be defined on top of \verb!$^eq()!
as necessary.

These functions, and the syntactic idioms that use them, are best understood in concrete examples.


\subsection{Contiguous Reduplication}

\cprotect\subsubsection{The \verb!$^redup()! Function}

In the majority of languages that exhibit reduplication, there is a single
reduplicant that appears contiguous with (next to) the substring being
copied, sometimes 
separated in the orthography by a special symbol, which is often the hyphen.

To model such contiguous-reduplication phenomena Kleene offers the \verb!$^redup($lang, $sep="")! function,
in which \verb!$lang! is a finite language of substrings to be reduplicated, and \verb!$sep!
denotes the separator symbol, which by default is the empty string.  Both \verb!$lang! and \verb!$sep! must
denote acceptors.

Before we look at the full idioms needed to model reduplication, let's see
what \verb!$^redup()! itself does:

\begin{Verbatim}
$lang = foo | bar | bas ;
$r = $^redup($lang, \-) ;
pr $r ;
\end{Verbatim}

\noindent
The output of this little script is

\begin{Verbatim}
foo-foo
bar-bar
bas-bas
\end{Verbatim}

Similary, the output of the following script

\begin{Verbatim}
$lang = foo | bar | bas ;
$r = $^redup($lang, "") ;  // or just $^redup($lang)
print $r ;
\end{Verbatim}

\noindent
is

\begin{Verbatim}
foofoo
barbar
basbas
\end{Verbatim}


\subsubsection{Full (Contiguous) Reduplication}

Indonesian illustrates productive full-root and full-stem reduplication, and the standard
orthography now dictates that a hyphen be used as the separator.  For example, where \emph{buku} is the
word for book, the overt plural is written \emph{buku-buku};\footnote{Reduplication is not used if
the word appears in a syntactic context that makes the plurality
obvious, e.g., when \emph{buku} is preceded by
the words for the plural cardinal numbers two, three, four, etc.} and where \emph{rumah} is the word for house,
the overt plural is \emph{rumah-rumah}.  Though this was once a very difficult
form of reduplication to handle in finite-state systems, it is quite easy using \verb!$^redup()!.


Let's assume that we want to create a transducer \verb!$indonesianNouns!
that relates \emph{buku} on the lower
side to \emph{buku[NOUN]} on the upper side, and \emph{buku-buku} on the
lower side to \emph{buku[NOUN][REDUP]} on
the upper side.  This requires surrounding the call to \verb!$^redup()!
with some idiomatic code, as in this example:

\begin{Verbatim}
$nouns = buku | rumah | anjing ;  // book, house, dog

$reduplicatedNouns = 
        "":.* $nouns 
        _o_
        $^redup($nouns, \-) & [^-]* $^ignore($nouns, \-) ;

$indonesianNouns = $nouns '[NOUN]':"" |
                   $reduplicatedNouns ('[NOUN]' '[REDUP]'):"" ;
\end{Verbatim}

\noindent
Let's take this line by line.  The definition of \verb!$nouns!

\begin{Verbatim}
$nouns = buku | rumah | anjing ;  // book, house, dog
\end{Verbatim}

\noindent
is a language consisting of strings that, when overtly plural, need to be reduplicated.
These words will be considered the baseforms.
The following line

\begin{Verbatim}
$reduplicatedNouns = 
        "":.* $nouns 
        _o_
        $^redup($nouns, \-) & [^-]* $^ignore($nouns, \-) ;
\end{Verbatim}

\noindent
is a composition of two \fsm{}s, the first being the \fst{} 

\begin{Verbatim}
"":.* $nouns
\end{Verbatim}

\noindent
which has the un-reduplicated nouns on the upper side (the baseforms), related to lower-side
strings that start with any number of any symbols, ending with the un-reduplicated nouns.
The second line 

\begin{Verbatim}
$^redup($nouns, \-) & [^-]* $^ignore($nouns, \-)
\end{Verbatim}

\noindent
denotes
an acceptor that is the intersection of \verb!$^redup($nouns, \-)!, which is the language
consisting of ``buku-buku,'' ``rumah-rumah'' and ``anjing-anging,'' with 
\verb![^-]* $^ignore($nouns, \-)!, the language of strings starting with any number of
non-hyphen symbols, and ending with our baseform-noun strings, but ignoring any hyphens.
In this case, the second line could be simplified a bit to

\begin{Verbatim}
$reduplicatedNouns = "":.* $nouns 
                     _o_
                     $^redup($nouns, \-) & [^-]* \- $nouns ;
\end{Verbatim}

\noindent
but the call to \verb!$^ignore()! is part of the more general idiom, as will be seen in subsequent
examples.

If you test the resulting \fst{} in the usual way


\begin{Verbatim}
test $indonesianNouns ;
\end{Verbatim}

\noindent
and enter, in the lower-side entry field, \emph{buku}, the output will
be \emph{buku[NOUN]}.  And if you enter
\emph{buku-buku}, the output will be \emph{buku[NOUN][REDUP]}.


\subsubsection{Partial Contiguous Prefixal Reduplication}

The following example, from M\r{a}ns Huldén, models a reduplication
phenomenon found in Warlpiri,
where the first prosodic foot of a word, defined by the pattern C V C? C? V (a consonant, a vowel,
zero to two consonants, and a second vowel) is reduplicated.  No separating
hyphen is used in the orthography.  Thus if the
baseform is \emph{pangurnu}, the reduplicated form is \emph{pangupangurnu}.  Because the
reduplication is attached to the front of the word, this is known as \verb!prefixal!
reduplication.

\begin{Verbatim}
// Hulden's example of partial prefixal reduplication 
// in Warlpiri.  Reduplication of the initial prosodic foot 
// of the form C V C? C? V

$lexicon = pangurnu | tiirlparnkaja | wantimi | pakarni ;
$C = p | ng | rn | rl | k | j | w | n | t | m ;
$V = [aeiou] ;

// the pattern of the prosodic foot
$prosodicFoot = $C $V $C? $C? $V ;

// extract out the prosodic stems (the substrings that need to
// be reduplicated)
$prosodicStem = $^lowerside($lexicon _o_ $prosodicFoot (.:"")*) ;

// calculate/construct the reduplicated strings
$rmorphology = 
	("":.)* $lexicon
    _o_
    $^redup($prosodicStem, \-) .* & [^-]* $^ignore($lexicon, \-)
    _o_
    \- -> "" ;

// union the normal lexicon with the reduplicated strings
$warlpiri = $lexicon '[UNMARKED]':"" | 
            $rmorphology '[REDUP]':"" ;
\end{Verbatim}

If you test the result, you will find that lower-side \iostr{wantimi} is
related to upper-side \iostr{wantimi[UNMARKED]}, and
lower-side \iostr{wantiwantimi} is related to upper-side
\iostr{wantimi[REDUP]}.  Draw the network, test the
strings in both directions, and walk through the example until you are comfortable with it.  This
illustrates the general idiom for computing prefixal reduplication, and the main change for other languages
with this phenomenon will be to redefine the prosodic foot (or syllable) that is appropriate.


\subsubsection{Partial Contiguous Suffixal Reduplication}

(KRB:  research the actual facts of Dakota reduplication.  This is a pseudo example for now.)
The Dakota language is reputed to have a reduplication where the final C C V of an
adjective root is reduplicated to mark the plural.  Because the copy attaches to the end of
the word, this is known as \emph{suffixal} reduplication.   The idiom for suffixal
reduplication is very similar to that for prefixal reduplication.  If you have trouble
typing the ʃ used in the example, simply replace it with S or \verb!'[S]'! for now.

\begin{Verbatim}
$C = [ptkbdghfsʃzwz] ;
$V = [aeiou] ;

$adj = haska | waʃte | mazdi | tuftu ;

$pattern = $C $C $V ;

$syl = $^outputProj($adj _o_ (.:"")* $pattern) ;

$redupMorphology = 
		$adj ("":.)*
        _o_
        $^ignore($adj, \-) [^-]* & .* $^redup($syl, \-)
        _o_
        \- -> "" ;

$dakota = $adj ('[ADJ]''[SG]'):"" |
          $redupMorphology('[ADJ]''[REDUP]'):"" ;
\end{Verbatim}

If you test the resulting \fst{} in the usual way, the lower-side
\iostr{haska} is related to the
upper-side \iostr{haska[ADJ][SG]}, and lower-side \iostr{haskaska} is related to the upper-side
\iostr{haska[ADJ][REDUP]}.  Other languages with suffixal reduplication can be handled similarly,
modifying the \verb!$pattern! as appropriate for the language.  In another language, for example,
the pattern might be \verb!$C $V $C!.

\subsubsection{Partial Contiguous Infixal Reduplication}

[KRB:  examples taken from Wikipedia article on reduplication.  Check
them when possible.]

Samoan displays a kind of infixal reduplication in which a penultimate
syllable of the form C V is reduplicated.  Thus the reduplicated form of
\emph{savali} is \emph{savavali}, and the reduplicated form of
\emph{alofa} is \emph{alolofa}.  The final syllable can be just V, so
the reduplicated form of \emph{tamaloa} is \emph{tamaloloa}.

\begin{Verbatim}
$C = [fglmnpstvhkr\'] ;
$V = [aeiou] ;

$lang = savali | alofa | tamaloa ;

// pattern of penultimate syllable that is reduplicated
$pat = $C $V ;

// extract the syllables to be reduplicated
$sylls = $^outputProj($lang
                      _o_
					  (.:"")* $pat ($C:"")? $V:"") ;

$rmorph = $lang
          _o_
          [^-]* "":(\- [^-]* \-)  $C? $V 
          _o_
          [^-]* $^redup($sylls, \-) \- $C? $V
          _o_
          \- -> "" ;

$samoan = $lang '[UNMARKED]':"" | $rmorph '[REDUP]':"" ;
\end{Verbatim}

\subsubsection{Full Reduplication with Alternations}

\citet[p. 57]{sproat:1992} lists the following examples of full-stem
prefixal reduplication in Javanese that also involve some vowel
alternations.\footnote{Sproat in turn
cites an unpublished manuscript by Paul
Kiparsky, ``The Phonology of Reduplication,'' Stanford University, that
I have not yet examined.}

\vspace{0.5cm}

\begin{tabular}{|l|l|l|}
\hline
     \textbf{Base}   & \textbf{Habitual-Repetitive} & \textbf{Gloss}\\
\hline
     adʊs   & odas+adʊs         &  `take a bath'\\
     bali   &bola+bali          & `return'\\
     bosən  &bosan+bosən        & `tired of'\\
     ɛlɛq   &elaq+ɛlɛq          & `bad'\\
     ibu    &iba+ibu            & `mother'\\
\hline
     dolan  &dolan+dolɛn       &  `engage in recreation'\\
     udan   &udan+udɛn         &  `rain'\\
\hline
     djaran &djoran-djaran     &  `horse'\\
     djaran &djoran-djɛrɛn     & `horse'\\
     djaran &djaran-djɛrɛn     &  `horse'\\
\hline
\end{tabular}

\vspace{0.5cm}

\noindent
Here is Sproat's description of the alternations:

\begin{quote}
As Kiparsky describes it, if the second vowel of the stem is not /a/---
 as in the first five examples above---the algorithm is as follows: 
 copy the stem, make the first vowel of the left copy nonlow, with the 
 result that /a/ is changed to /o/ and /ɛ/ is changed to /e/, and change 
 the second vowel to /a/. If the second vowel of the stem is /a/, then the 
 left copy remains unchanged but the /a/ in the right copy changes to /ɛ/. 
 If both vowels are /a/, as in djaran `horse', there are three different 
 possibilities, as shown [above].
\end{quote}

M\r{a}ns Huldén wrote a Foma script to handle the Javanese examples, as
just described, and here, with his permission, is his script translated into Kleene syntax:

\begin{Verbatim}
$Base = adʊs | bali | bosən | ɛlɛq | ibu | 
        dolan | udan | djaran ;
$V = [aeiouɛʊə] ;
$C = [bdfgjklmnpqrstvwx] ;
$Lexicon =  "" <- \- .* /  _ '[HabRep]'
            _o_
            $^redup($Base, \-) '[HabRep]':"" | 
			$Base '[Unmarked]':"" ;

// The Lexicon transducer now contains input-output
// pairs such as:

// b a l i           [HabRep]    (input)     
// b a l i - b a l i             (output)         
// or
// b a l i [Unmarked]   (input)
// b a l i              (output)


// This we can now compose with the rules that change
// vowels in the two copies yielding, e.g., 
// bali-bali -> bola-bali, etc.

// The main rule
$Rule1 =  
    $^parallel( $a -> $b / # $C* _ $C* ($V - a) .* \-
               { where $a _E_ $@(a, ɛ), 
                       $b _E_ $@(o, e) },
               $V -> a / # $C* $V $C* _ .* \- ,
               a -> ɛ / \- $C* ($V - a) $C* _ .*
             ) ;

// To handle the "three different possibilities" 
// seen in "djaran", where both vowels are 'a'
$Rule2 =  
  $^parallel(
    a:(o|a) ($C|$V)* \- $C* a:ɛ $C* a:ɛ -> /  #   $C* _ .* ,
    a ->?  o /  #   $C* _ .* \-
  ) ;

$javanese = $Lexicon _o_ $Rule1 _o_ $Rule2 ;
\end{Verbatim}

\noindent
The resulting \verb!$javanese! \fst{} can be tested in the usual way.

\subsection{Discontiguous Reduplication}

\cprotect\subsubsection{The \verb!$^eq()! Function}

The \verb!$redup($lang, $sep="")! function demonstrated above calls a lower-level
function \verb!$^eq()! that is harder to use but permits the modeling of
discontiguous reduplication.

[KRB:  Fill out this section]


\section{Semitic Interdigitation}

This section is currently empty.

\section{Constraining Long-distance Dependencies}

This section is currently empty.
