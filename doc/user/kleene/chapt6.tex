\chapter{Lists}


\section{Lists of \fsm{}s and Numbers}

In addition to individual \fsm{}s, and individual numbers,
Kleene also supports collections of \fsm{}s and collections of numbers.
These collections are implemented as linked lists,\footnote{In the
underlying Java code, they are implemented as LinkedList objects.} and
functions are provided to manipulate them as lists, stacks, queues and
double-ended queues (also known as deques, pronounced \emph{decks}).
They will be referred to collectively herein as lists.

Kleene lists must be homogeneous; that is, a single list must contain
only \fsm{}s, or only numbers, though the numbers may be a mix of
integers\footnote{Kleene integers are always stored internally as Java
Long objects.} and floats.\footnote{Kleene floats are always stored
internally as Java Double objects.}  Kleene lists are sometimes referred
to informally as arrays, but they are technically lists and are better
thought of as such.\footnote{Readers familiar with Lisp and Scheme are
warned that Kleene lists cannot contain lists.}

Kleene provides syntax to represent, assign, access and manipulate lists.
Lists can be passed as arguments to functions, and functions can
return a list as a result.  A \texttt{foreach} statement supports
iteration through the members of a list.

The sigil \verb!$@! marks lists of \fsm{}s, and \verb!#@! marks lists of
numbers. 

\section{List Literals, Identifiers, and Assignment}

List literals, also known as anonymous lists, are denoted by the
appropriate sigil, \verb!$@! for \fsm{}s and \verb!#@! for numbers, followed
by a parenthesized list of elements.

\begin{Verbatim}
$@(dog, cat, a, a*b+[c-g])  // a literal list of 4 FSMs 

#@(12, -45, 2.47, 0.326, 0) // a literal list of 5 numbers
\end{Verbatim}

\noindent
An identifier of the form \verb!$@!\emph{name} can be bound to an
\fsm{}-list value; and, in parallel, an identifier of the form
\verb!#@!\emph{name} can be bound to a number-list value.

\begin{Verbatim}
$@foo = $@(a, b, $fsm1, $fsm2, (dog|cat|elephant)s?) ;

#@bar = #@(1, 2, 12.23, 9, -234) ;
\end{Verbatim}

\noindent
When such assignment statements are executed in the \acro{gui}, the lists are
represented by named icons that appear automatically in the symbol-table window.

The \texttt{info} and \texttt{delete} commands work with list identifiers
just as they do with individual \fsm{} and number identifiers.

\begin{Verbatim}
info $@foo ;
info #@bar ;

delete $@foo ;
delete #@bar ;
\end{Verbatim}

\noindent
In the \acro{gui}, these commands can be accessed by right-clicking on a list
icon.

\section{Pre-Defined Functions Operating on Lists}

\subsubsection{Functions Accessing the Elements of a List} 
Individual list
elements can be accessed non-destructively from a list using the
following pre-defined functions.  Note that index counting starts at 0
(zero).

\vspace{.5cm}

\noindent
\begin{tabular}{|l|l|}
\hline
\verb!$^head($@list)! & Returns the zeroth FSM element of the argument list\\
\hline
\verb!$^getLast($@list)! & Returns the last FSM element of the argument list\\
\hline
\verb!$^get($@list, #n)! & Returns the nth FSM element of the argument list\\
\hline
\end{tabular}

\vspace{.5cm}

\noindent
Such functions return \fsm{}s and can be called in the usual ways, 
e.g.

\begin{Verbatim}
$@mylist = $@(a, b, c, d) ;
$fsm = $^head($@mylist) ;
\end{Verbatim}

\noindent
would set \verb!$fsm! to the \fsm{} denoted by \texttt{a}.

Parallel functions are defined for number-lists:

\vspace{.5cm}

\noindent
\begin{tabular}{|l|l|}
\hline
\verb!#^head(#@list)! & Returns the zeroth number element of the argument list\\
\hline
\verb!#^getLast(#@list)! & Returns the last number element of the argument list\\
\hline
\verb!#^get(#@list, #n)! & Returns the nth number element of the argument list\\
\hline
\end{tabular}

\vspace{.5cm}

The following functions return a new list containing all or part of the
elements of the argument list.  These operations are non-destructive.

\vspace{.5cm}

\noindent
\begin{tabular}{|l|p{4.5cm}|}
\hline
\verb!$@^copy($@list)! & Returns a shallow copy of the argument FSM-list\\
\hline
\verb!$@^tail($@list)! & Returns a new list containing all but the first element of the
argument FSM-list\\
\hline
\verb!$@^getSlice($@list, #n, #r:#s, ...)! & Returns a new list containing the
indicated elements of the argument FSM-list\\
\hline
\end{tabular}

\vspace{.5cm}

\noindent
\begin{tabular}{|l|p{4.5cm}|}
\hline
\verb!#@^copy(#@list)! & Returns a shallow copy of the argument number-list\\
\hline
\verb!#@^tail(#@list)! & Returns a new list containing all but the first element of the
argument number-list\\
\hline
\verb!#@^getSlice(#@list, #n, #r:#s, ...)! & Returns a new list containing the
indicated elements of the argument number-list\\
\hline
\end{tabular}

\vspace{.5cm}

\noindent
The statement

\begin{Verbatim}
$@tailList = $@^tail($@(a, b, c)) ;
\end{Verbatim}

\noindent
sets \verb!$@tailList! to the value \verb!$@(b, c)!.  As in Scheme,
Haskell and Scala, the \texttt{head} and \texttt{tail} functions throw an
exception if the argument is an empty list.  The call

\begin{Verbatim}
$@newList = $@^getSlice($@list, 2, 4, 7:10) ;
\end{Verbatim}

\noindent
sets \verb!$@newList! to a list containing items 2, 4, 7, 8 and 9 of
\verb!$@list!.  Note that index counting starts at 0, and the notation
\emph{n}:\emph{m} is inclusive of \emph{n} but exclusive of \emph{m}, so
\texttt{7:10} includes 7, 8 and 9.

The following functions destructively return an individual element from
the argument list, removing that element from the list.  As elsewhere in
Kleene, destructive function names are marked with a final \verb+!+.

\vspace{.5cm}

\noindent
\begin{tabular}{|l|p{6cm}|}
\hline
\verb+$^pop!($@list)+ & Removes and returns the zeroth element of a FSM-list\\
\hline
\verb+$^removeLast!($@list)+ & Removes and returns the last element of a
FSM-list\\
\hline
\verb+$^remove!($@list, #n)+ & Removes and returns the nth element of a
FSM-list\\
\hline
\end{tabular}

\vspace{.5cm}

\noindent
\begin{tabular}{|l|p{6cm}|}
\hline
\verb+#^pop!(#@list)+ & Removes and returns the zeroth element of a number-list\\
\hline
\verb+#^removeLast!(#@list)+ & Removes and returns the last element of a number-list\\
\hline
\verb+#^remove!(#@list, #n)+ & Removes and returns the nth element of a number-list\\
\hline
\end{tabular}

\vspace{.5cm}

The alias \verb+$^pop_back!($@list)+ is pre-defined for
\verb+$^removeLast!($@list)+, and  \verb+#^pop_back!(#@list)+ is
pre-defined as an alias for \verb+#^removeLast!(#@list)+.

The following functions add elements destructively to a list, changing
the list, and returning the changed list.

\vspace{.5cm}

\noindent
\begin{tabular}{|l|p{6.5cm}|}
\hline
\verb+$@^push!($fst, $@list)+ & Push \verb!$fst! on the front of the FSM-list, and return the
modified list\\
\hline
\verb+$@^add!($@list, $fst)+ & Add \verb!$fst! on the end of the FSM-list, and return the modified
list\\
\hline
\verb+$@^addAt!($@list, #n, $fst)+ & Add/insert \verb!$fst! at the indicated index of
the FSM-list, and return the
modified list\\
\hline
\end{tabular}

\vspace{.5cm}

\noindent
\begin{tabular}{|l|p{6.5cm}|}
\hline
\verb+#@^push!(#num, #@list)+ & Push \verb!#num! on the front of the number-list, and return the
modified list\\
\hline
\verb+#@^add!(#@list, #num)+ & Add \verb!#num! on the end of the number-list, and return the modified
list\\
\hline
\verb+#@^addAt!(#@list, #n, #num)+ & Add/insert \verb!#num! at the indicated index of
the number-list, and return the
modified list\\
\hline
\end{tabular}

\vspace{.5cm}

The alias \verb+$@^push_back!($@list, $fst)+ is pre-defined for the function
\verb+$@^add!($@list, $fst)+, and \verb+#@^push_back!(#@list, #num)+ is pre-defined as an
alias for \verb+#@^add!(#@list, #num)+.

The \verb+$@^set!($@list, #n, $fst)+ function and the parallel function for
number-lists
\verb+#@^set!(#@list, #n, #num)+ reset the list value at index
\verb!#n! to \verb!$fst! and \verb!$num!, respectively, and return the modified list.

\vspace{.5cm}

\noindent
\begin{tabular}{|l|p{5.5cm}|}
\hline
\verb+$@^set!($@list, #n, $fst)+ & Add/insert \verb!$fst! at the indicated index, and return the
modified list\\
\hline
\verb+#@^set!(#@list, #n, #num)+ & Add/insert \verb!#num! at the indicated index, and return the
modified list\\
\hline
\end{tabular}

\vspace{.5cm}

\subsection{Functions Joining the Elements of a List}

The function 
\verb!$^reduceLeft($^bin, $@list)! takes as arguments a binary function
\verb!$^bin! that takes two \fsm{} arguments and a list of \fsm{}s
\verb!$@list!; and it returns an \fsm{} value.
\verb!$^reduceLeft($^bin, $@list)!  
applies the passed-in binary function to combine the members of
the list from left to right and returns the result.  The behavior is best
seen in the following example, where a list of \fsm{}s is reduced first
by concatenation and then by composition.

\begin{samepage}
\begin{Verbatim}
$^concat($x, $y)  { return $x $y; }
$^compose($x, $y) { return $x _o_ $y ; }

$@list = $@(a:b, b:c, c:d) ;

$catfsm = $^reduceLeft($^concat, $@list) ;
// $catfsm encodes the relation  a:b b:c c:d

$compfsm = $^reduceLeft($^compose, $@list) ;
// $compfsm encodes the relation a:d
\end{Verbatim}
\end{samepage}

\verb!$^foldLeft($^bin, $@list, $init)! is like \verb!$^reduceLeft()! except that it has
three arguments: a binary function, an \fsm{} list and an initial \fsm{} value.
\verb!$^foldLeft()! is often preferable to \verb!$^reduceLeft()! because
it returns the initial value, rather than throwing an exception, when
the argument list is empty.

\begin{Verbatim}
$result = $^foldLeft($^concat, $@list, "") ;
\end{Verbatim}

Similarly, the numerical function 
\verb!#^reduceLeft(#^bin, #@list)! 
takes as arguments 
a binary function \verb!#^bin! and
a list of numbers \verb!#@list!,
and it returns a number value.  In
the following example, a list of numbers is reduced first by addition and then by
multiplication.

\begin{Verbatim}
// define some binary functions
#^add(#x, #y)  { return #x + #y ; }
#^mult(#x, #y) { return #x * #y ; }

#@list = #@(1, 2, 3, 4) ;

#sum = #^reduceLeft(#^add, #@list) ;
// #sum has the value 10

#product = #^reduceLeft(#^mult, #@list) ;
// #product has the value 24
\end{Verbatim}

\verb!#^foldLeft(#^bin, #@list, #init)! is like \verb!#^reduceLeft()! except that it
has three arguments: a binary function, a number list and an initial number value.
\verb!#^foldLeft()! is often preferable to \verb!#^reduceLeft()! because
it returns the initial value, rather than throwing an exception, when
the argument list is empty.

With the \texttt{foldLeft} and \texttt{reduceLeft} functions, it is sometimes convenient to pass an
anonymous function as the first argument, e.g.

\begin{Verbatim}
// calling foldLeft with anonymous functions
#sum     = #^foldLeft(#^(#a, #b) {return #a + #b;}, #@list, 0) ;
#product = #^foldLeft(#^(#a, #b) {return #a * #b;}, #@list, 1) ;
\end{Verbatim}


\subsection{Functions Returning the Map of a List}

The \verb!$@^map($^mon, $@list)!
applies the single-argument function \verb!$^mon! to each \fsm{} element of
\verb!$@list! and returns a list of the returned \fsm{} values.  The arguments
can be anonymous.  Note that the \verb!$@^map($^mon, $@list)! function has the sigil
\verb!$@^!, indicating a function (function names are always immediately preceded by \verb!^!) that returns a list of \fsm{}s (\verb!$@!).

\begin{Verbatim}
$@inList = $@(a, b, c, d) ;

// called with an anonymous function that concatenates
//    the two arguments
$@outList = $@^map($^($fst){ return $fst $fst ; }, $@inList) ;
// $@outList will be $@(aa, bb, cc, dd)

// called with an anonymous function and an anonymous list
$@outList = $@^map($^($fst){ return $fst $fst ; }, 
                   $@(a, b, c, d)) ;
\end{Verbatim}

Similarly, the 
\verb!#@^map(#^mon, #@list)!  applies the single-argument 
function \verb!#^mon! to each number
element of \verb!#@list! and returns a list of the returned values.  The
arguments can be anonymous.  Note that the \verb!#@^map(#^mon, #@list)! function has the sigil
\verb!#@^!, indicating a function (function names are always immediately preceded by
\verb!^!) that returns a list of numbers (\verb!#@!).

\begin{Verbatim}
#@inList = #@(0, 1, 2, 3) ;

// called with an anonymous function
#@outList = #@^map(#^(#n){ return #n + #n ; }, #@inList) ;
// #@outList will be $@(0, 2, 4, 6)

// called with an anonymous function and an anonymous list
#@outList = #@^map(#^(#n) { return #n * #n ; }, #@(1, 2, 3, 4) ) ;
// #@outlist will be $@(1, 4, 9, 16)
\end{Verbatim}

\subsection{Functions Returning the Sigma of an \fsm{}}

The \verb+$@^getSigma!($fst)+ function returns the sigma of the argument
\fsm{} as a list of \fsm{}s, each one consisting of a start state and a final
state, linked by one arc labeled with a symbol from the sigma.  Special symbols
used only internally in Kleene are excluded from the sigma.

The parallel \verb+#@^getSigma!($fst)+ function returns the sigma of the argument
\fsm{} as a list of integers.

\begin{Verbatim}
$fsm = abc ;

$@list = $@^getSigma($fsm) ;
#@list = #@^getSigma($fsm) ;
\end{Verbatim}

\subsection{Functions Returning the Size of a List}

The \verb+#^size(+\textit{list}\verb!)! function takes a list argument,
either an \fsm{}-list or a number-list, and returns the
size as an integer.\footnote{Kleene does not, in general, support function
overloading, wherein multiple functions of the same name are distinguished
by the type and/or number of arguments they take.
The \texttt{size} pseudo function is an exception,
wired into the parser and specially interpreted to accept an argument of either list
type.}

\begin{Verbatim}
#n = #^size(#@(1, 2, 3, 4)) ;

#m = #^size($@(a, b, c, d, e, f)) ;
\end{Verbatim}

The \verb!#^isEmpty(!\emph{list}\verb!)! function takes a list argument and
returns 1 (true) if the list is empty and 0 (false) otherwise.

\section{Iteration through Members of a List}


The \texttt{foreach} statement iterates through the elements of a list, allowing
some operation or operations to be performed on each element.  The body of
the \texttt{foreach} statement is a block or a single statement.

\begin{Verbatim}
foreach ($fsm in $@list) {
    info $fsm ;
}

foreach ($fsm in $@list) info $fsm ;

foreach (#num in #@list) {
    info #num ;
}

foreach (#num in #@list) info #num ;
\end{Verbatim}

\noindent
As a more concrete example, consider

\begin{Verbatim}
$fsm = abc ;

foreach (#cpv in #@^getSigma($fsm)) info #cpv ;
// output:
// Long value: 98
// Long value: 99
// Long value: 97
\end{Verbatim}

For each iteration of the block or statement, the iteration identifier,
\verb!$fsm!, \verb!$num! and \verb!#cpv! in these examples, is bound, in
the current frame, to the next item of the list, and after the
\texttt{foreach} statement, the iteration identifier is left bound to the
last value in the list.\footnote{The \texttt{foreach} statement does not,
like a function call, trigger the allocation of a new frame.}  In the
following example, the value of \verb!#num! at Point B is 3.

\begin{Verbatim}
#num = 0 ;      // Point A
foreach (#num in #@(1, 2, 3)) {
	print #num ;
}
info #num ;    // Point B
\end{Verbatim}

(KRB:  consider the wisdom of treating the iteration identifier this
way.  Perhaps it should be (at least conceptually, a new identifier that
is bound for each value, and perhaps it should disappear after the
foreach loop is finished.)

\section{User-Defined Functions and Lists}


Users can define their own functions that take lists as arguments, and functions
that return lists, e.g.\@

\begin{Verbatim}
$@^reverse($@list) {
    $@result = $@() ;  // an empty list
    foreach ($fsm in $@list) {
        $@result = $@^push!($fsm, $@result) ;
    }
    return $@result ;
}

$@reversedList = $@^reverse($@(a, b, c)) ;
\end{Verbatim}


\section{Traditional Array-indexing Syntax}


[KRB:  This section is work-in-progress.]

Although the current implementation of lists provides access functions like
\verb!$^get($@list, #n)!, to retrieve a value at a specified index, 
some users might prefer to use postfixed indices in
square brackets or parentheses, as in most implementations of arrays.

\begin{Verbatim}
// NOT implemented in Kleene
$@arr[0]   // array indexing as in C/C++, Java
$@arr(0)   // array indexing as in Scala
\end{Verbatim}

\noindent
Assuming for a second that such post-fixed indexing is desired, it is not yet
clear that it could be implemented easily in Kleene.
Because square brackets normally denote symbol unions in Kleene regular
expressions, tokenizing and
parsing them differently when they are intended to indicate an index would be a
challenge; the notation with parentheses, as used in Scala, might be easier to
implement and
less likely to cause confusion.  As is often the case, tokenization is a bigger
challenge than the parsing.

If the expression denoting a list is just an
identifier like \verb!$@list!, as in the examples above, the challenge is trivial.  But if the
expression denoting a list is more complex, such as a function call returning a list,
and having arbitrary arguments to tokenize and parse, then correctly tokenizing a
post-fixed index expression that uses square brackets would be rather involved.
Again, using parentheses might avoid most of the problem.

Traditional arrays also allow setting of elements of the array, e.g.

\begin{Verbatim}
$@arr[2] = a*b+[c-g]? ;
\end{Verbatim}

\noindent
and this would also present challenges for Kleene tokenization and parsing.

At this point, we should continue to think about post-fixed indexing
syntax.  As Kleene lists are not really arrays, and as \texttt{foreach}
statements are provided for iterating easily through lists, I believe
that we can dispense with postfix iteration altogether.


