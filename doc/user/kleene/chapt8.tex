\chapter{Arithmetic Expressions}

\label{chapt:arithmeticexpressions}

\section{Mindtuning for Arithmetic Expressions}

While \Kleene{} is designed primarily for creating and manipulating
finite-state networks, it does support arithmetic expressions, variables
that hold arithmetic values, functions that return arithmetic values,
etc.  Variables with an arithmetic value are marked with a \# sigil,
functions that return an arithmetic value with \verb!#^!, etc.  Wherever a
simple integer or float can appear in \Kleene{} syntax, including
numbered iterations like \verb!a{2,4}! and weights like \verb!<0.1>!, an
arbitrarily complex arithmetic expression can appear, e.g.\@ 
\verb!a{2, #maxlength - 1}! and \verb!<#defaultweight + .01>!.\footnote{Internally,
\Kleene{} stores arithmetic values as either a Long or a Double object.}

The arguments passed to a function could be any mix of regular
expressions and arithmetic expressions, and one of the biggest challenges
during \Kleene{} design and development was the distinguishing and proper
tokenization/parsing of the two separate expression types.  For example,
both use the plus sign as an operator, but in arithmetic expressions it
is a binary infix operator of fairly low precedence, e.g.\@  \verb!2+3!,
while in regular expressions it is a unary postfix operator of fairly
high precedence, e.g.\@ \verb!ab+c!.  Distinguishing the two
expression types by inventing new operators---either for regular
expressions or for arithmetic expressions---was judged to be
completely unacceptable; and forcing users to surround regular
expressions, or arithmetic expressions, with some kind of explicit
delimiters, such as the Perl slashes /\ldots/, was deemed inelegant
and undesirable.

The solution adopted was to define a systematic set of sigils starting
with \verb!$! for network-value variables and functions, and \verb!#! for
arithmetic-value variables and functions.  Parser lookahead distinguishes
\verb!#a+#b! as an arithmetic expression, involving addition, from
\verb!$a+$b!, which is a regular expression indicating the concatenation
of one or more iterations of network \verb!$a! with network \verb!$b!.
Once the sigil system is mastered, users can, in almost all cases, simply
type familiar regular and arithmetic expressions in appropriate places. 

The remaining problematic cases are expressions like \verb!2! and
\verb!2+3!, which start with digits.  Are they arithmetic expressions,
having a integer or float value, or regular expressions, having a network
value?  The \Kleene{} solution is to treat bare digits by default as
arithmetic expressions.  To be interpreted as literal characters, and
therefore regular expressions, digits must be literalized in the usual
\Kleene{} ways:

\begin{itemize}
\item
Using the prefix backslash literalizer:  \verb!\2!
\item
Surrounding them with double quotes: \verb!"2"!, or
\item
Putting them in square-bracketed expressions: \verb![2]! \verb![^2]! \verb![0-9]! \verb![^0-9]!
\end{itemize}

\section{Primary Arithmetic Expressions}

The primary arithmetic expressions are the following:

\vspace{.5cm}

\renewcommand\tabcolsep{1.25mm}

\noindent
\begin{tabular}{|p{5.5cm}|l|}
\hline
\verb!0 1 42!  & decimal integers\\
\hline
\verb!0x1 0X1 0x1A 0X23ABC! & hex integers\\
\hline
\verb!2.5 .234 103.! & decimal floats\\
\hline
\verb!#n #count #foo! & names of variables denoting an arithmetic value\\
\hline
\verb!#^abs(#numexp)! \verb!#^arcCount($regexp)!
\verb!#^stateCount($regexp)! \verb!#^myfunction(!\emph{args}\verb!)! & names of functions returning an arithmetic value\\
%\hline
%\verb!#@arr[1]! & arithmetic array reference\\
\hline
\end{tabular}

\vspace{.5cm}

\noindent
Note that \Kleene{} does not support octal (base 8) representations of integers.

\section{Arithmetic Expression Operators}

The following regular expression operators are available, 
listed from high to low precedence.

\vspace{0.5cm}

\noindent
\begin{tabular}{|l|l|l|l|}
\hline
( ) &  parenthetical grouping & circumfix &\\
\hline
\verb!+ -! & positive, negative & prefix &\\
\hline
\verb!* / %! & multiply, divide, mod & infix & left assoc.\\
\hline
\verb!+ -!  & add/subtract & infix & left assoc.\\
\hline
\verb/< <= >= == != >/ & Boolean comparisons & infix & left assoc.\\
\hline
\verb/!/ & Boolean NOT  & prefix & \\
\hline
\verb!&&! & Boolean AND & infix & left assoc.\\
\hline
\verb!||! & Boolean OR  & infix & left assoc.\\
\hline
\end{tabular}

\vspace{0.5cm}

\noindent
The result of boolean operations is always 1 (true) or 0 (false).  There
is no separate boolean type in Kleene.

\section{Arithmetic Functions}

The following pre-defined arithmetic-valued functions are provided.

\vspace{0.5cm}

\noindent
\begin{tabular}{|l|}
\hline
\verb!#^pathCount($fst)! \\
\verb!#^stateCount($fst)! \\
\verb!#^arcCount($fst)! \\
\verb!#^arity($fst)! \\
\hline
\verb!#^abs(#num)! \\
\verb!#^ceil(#num)! \\
\verb!#^floor(#num)! \\
\verb!#^round(#num)! \\
\verb!#^log(#num)! \\
\verb!#^prob2c(#prob)! \\
\verb!#^pct2c(#pct)! \\
\hline
\verb!#^long(#num)! or \verb!#^int(#num)! \\
\verb!#^double(#num)! or \verb!#^float(#num)! \\
\verb!#^rint(#num)! \\
\hline
\end{tabular}

\vspace{0.5cm}

In Kleene there is no difference between \verb!#^long(#num)! and
\verb!#^int(#num)!; both are stored as long integers.\footnote{Long
integers are stored internally as Java Long objects.}  Similarly, there
is no difference between \verb!#^double(#num)! and \verb!#^float(#num)!;
both are stored as double values.\footnote{Double values are stored
internally as Java Double objects.}  

The \verb!#^log(#num)! function returns the natural logarithm of the
argument.
The \verb!#^prob2c(#prob)! takes an argument that represents a probability,
where probabilities range from 1.0 (certain) to 0.0 (impossible), and
returns a cost weight suitable for the Tropical Semiring.\footnote{In
	the Tropical Semiring, the weights are \emph{costs}.  For any
probability \emph{p}, the corresponding cost is $-$log(p).  A probability
of 1.0 corresponds to a cost of 0.0, and a probability of 0.0
corresponds to an infinite cost.}  The \verb!#^prob2c(#prob)! function might
be used in examples like the following,

\begin{Verbatim}[fontsize=\small]
$fsm = a b ( c <#^prob2c(0.1)> | d <#^prob2c(0.9)> ) ;
\end{Verbatim}

\noindent
where the path containing \emph{c} has 0.1 probability, and the path containing
\emph{d} has 0.9 probability.  The \verb!#^pct2c(#pct)! is similar but allows you
to express the weights as percentages, e.g.\ 10\% vs.\ 90\%.  The following example is equivalent to the
one just above.

\begin{Verbatim}[fontsize=\small]
$fsm = a b ( c <#^pct2c(10)> | d <#^pct2c(90)> ) ;
\end{Verbatim}


[KRB:  explain why the \verb!#^pathCount()! is not completely accurate.
Explain the other functions..]  

\section{Boolean Functions}

\subsection{Special Case of Arithmetic Functions}

Boolean functions are special cases of arithmetic functions that return 1 for true and 0 for false.  They
are typically used in if-else expressions or in loops for testing qualities of finite-state machines.
The predefined boolean functions are 

\vspace{0.5cm}

\noindent
\begin{tabular}{|l|}
\hline
\verb!#^isAcceptor($fst)! \\
\verb!#^isTransducer($fst)! \\
\verb!#^isRtn($fst)! \\
\hline
\verb!#^isWeighted($fst)! \\
\verb!#^isIDeterministic($fst)! or \verb!#^isUDeterministic($fst)!\\
\verb!#^isODeterministic($fst)! or \verb!#^isLDeterministic($fst)!\\
\verb!#^isCyclic($fst)!\\
\verb!#^isIBounded($fst)! or \verb!#^isUBounded($fst)!\\
\verb!#^isOBounded($fst)! or \verb!#^isLBounded($fst)!\\
\verb!#^containsOther($fst)! \\
\verb!#^hasClosedAlphabet($fst)! \\
\verb!#^isEpsilonFree($fst)! \\
\hline
\verb!#^isEmptyLanguage($fst)! \\
\verb!#^isEmptyStringLanguage($fst)! \\
\verb!#^isString($fst)! or \verb!#^isSingleStringLanguage($fst)!\\
\verb!#^containsEmptyString($fst)! \\
\verb!#^isUniversalLanguage($fst)! \\
\hline
\end{tabular}

\vspace{0.5cm}
\noindent
and others will no doubt be defined as they become needed.

The \verb!#^isAcceptor($fst)! function returns true if and only if the
argument network is semantically an
acceptor.  In OpenFst, where all arc labels \emph{i}:\emph{o} in a finite-state machine
are two-level, with an input label \emph{i} and an output label \emph{o}, an acceptor is a special case of a
machine wherein each arc-label mapping is an identity mapping.\footnote{If the finite-state machine contains an arc labeled
\acro{other\_nonid:other\_nonid}, then it will look like an acceptor to the OpenFst library, but in Kleene such
a network is semantically and in practice a transducer; if \verb!#^isAcceptor()! is called on such
a network the return value is 0 (false).  Conversely, if
\verb!#^isTransducer()! is called on any network that
contains an arc labeled \acro{other\_nonid:other\_nonid}, the return value is 1 (true).}

The \verb!#^isTransducer($fst)! function returns true if and only if the argument network is semantically a
non-identity transducer.  Thus if the machine contains any arc label like a:b, where the upper label is
different from the lower label, \verb!#^isTransducer! will return true.\footnote{If the argument contains an
arc labeled 
\acro{other\_nonid:other\_nonid},
then it will look like an identity relation to OpenFst itself, but in
Kleene this is semantically a non-identity transducer and \verb!#^isTransducer! will return 0 (false).}

The \verb!#^isWeighted($fst)! function returns true if and only if the argument
finite-state machine contains any arc
with a non-neutral weight value.

The \verb!#^isIDeterministic($fst)! function returns true if and only if the finite-state machine is \emph{input}-side
deterministic, where ``input'' here is taken in the OpenFst sense of the upper side.
Similarly, the function \verb!#^isODeterministic($fst)! returns true if and only if the
network is \emph{output}-side deterministic, in the OpenFst sense.\footnote{The spellings IDeterministic and ODeterministic are
used inside the OpenFst library.}

The \verb!#^isCyclic($fst)! function returns true if and only if the finite-state machine has cycles, also
known as loops.

The \verb!#^isIBounded($fst)! function returns true if and only if the finite-state machine has no input side
epsilon loops.

The \verb!$^isOBounded($fst)! function returns true if and only if the finite-state machine has no output side
epsilon loops.

The \verb!#^containsOther($fst)! function returns true if and only if the network
contains at least one \acro{other\_id} or \acro{other\_nonid} label.
Conversely, 
\verb!#^hasClosedAlphabet($fst)! returns true if and only if the
alphabet of the network is closed, i.e.\@ the network does not contain any 
\acro{other\_id} or \acro{other\_nonid} labels.  Thus
\verb!#^hasClosedAlphabet($fst)! is equivalent to 
\verb@!#^containsOther($fst)@.

The \verb!#^isEpsilonFree($fst)! function returns true if and only if the network contains no
\acro{epsilon:epsilon} arcs,
i.e.\@ no double-sided epsilon labels.  Because Kleene routinely runs epsilon-removal
on all networks, which removes such double-sided epsilon arcs, unless such removal is
explicitly turned off, there is seldom any practical need to call 
\verb!#^isEpsilonFree($fst)!.

The \verb!#^isEmptyLanguage($fst)! function returns true if and only if the network encodes the empty
language, i.e.\@ the language that contains no strings at all, not even the empty string.

The \verb!#^isEmptyStringLanguage($fst)! function returns true if and only if the network encodes the
empty-string language, i.e.\@ the language that contains only the empty string.

The \verb!#^isString($fst)! function, aliased as \verb!#^isSingleStringLanguage($fst)!, returns true if
and only if the network encodes a language of exactly one string.

The \verb!#^containsEmptyString($fst)! function returns true if and only if the network encodes a
language that contains the empty string.

The \verb!#^isUniversalLanguage($fst)! function returns true if and only if the network encodes the
universal language, i.e.\@ the language that contains all possible strings.  [KRB, 2012-08-19:  This
function will not be fully reliable until a ``compact sigma'' function is implemented.]

\subsection{Assert and Require Statements}

The \verb!assert(#numexp, $regexp)! statement has a required arithmetic first
argument, which is analyzed as a boolean, and an optional second
regular-expression argument, which must denote a single-string language.  If
the first argument analyzes as true, Kleene takes no further action and control simply passes to the next
statement.  But if the first argument analyzes as false, a runtime exception is thrown and
the single-string second argument, if present, is used as the exception
message.  The \verb!assert(#numexp, $regexp)! statement is typically used
for testing.

\begin{Verbatim}[fontsize=\small]
$fst = a - a ;
assert(#^isEmptyLanguage($fst), 
        "Should denote the empty language.") ;

$fst2 = a* ;
assert(#^containsEmptyString($fst2), 
        "Should contain the empty string.") ;
\end{Verbatim}

In complete parallel, the \verb!require(#numexp, $regexp)! statement also has a required arithmetic first
argument, which is analyzed as a boolean, and an optional second
regular-expression argument, which must denote a single-string language.  If
the first argument analyzes as true, Kleene takes no further action and control simply passes to the next
statement.  But if the first argument analyzes as false, a runtime exception is thrown and
the single-string second argument, if present, is used as the exception
message.  The \verb!require(#numexp, $regexp)! statement is typically used
in user-defined functions to impose semantic restrictions on the
arguments.\footnote{The need for separate \verb!assert(#numexp, $regexp)!
and \verb!require(#numexp, $regexp)! arguments could be debated.  One
possibility would be to implement a global switch that would cause all
\verb!assert(#numexp, $regexp)! statements, but not 
\verb!require(#numexp, $regexp)! statements, to be ignored at runtime.}

\begin{Verbatim}[fontsize=\small]
#^myIntersect($a, $b) {
	require(#^isAcceptor($a) && #^isAcceptor($b), 
    "The arguments to #^myIntersect must denote an acceptor.") ;

	return $a & $b ;
}
\end{Verbatim}


