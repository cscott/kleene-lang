
\chapter{Other Syntax}

\section{Void Functions with Names}

Void functions return no value, and their names are
prefixed with the plain \verb!^!
sigil, like all Kleene functions, but with nothing before the \verb!^!.
Such void functions can be used to abbreviate and generalize a
sequence of Kleene commands.

\begin{Verbatim}
^inputSigma($fst) {
	$input = $^inputProj($fst) ;
	sigma $input ;
	draw $input ;
}

// an invocation of the defined void function
^inputSigma(abc) ;
\end{Verbatim}

\section{Anonymous Void Functions}

An anonymous void function looks like
\verb!^(!\ldots\emph{args}\ldots\verb!)!\verb!{!\ldots\emph{body}\ldots\verb!}!,
with the plain \verb!^! sigil
marking a function that has no return value. 


\section{Control Syntax}

In addition to \fsm{}-, arithmetic- and function-assignment statements,
\Kleene{} provides \verb!if-elsif-then! statements, \verb!while!  and
\verb!until!  loops, iteration over list elements, and a variety of
``housekeeping'' statements for input-output, executing pre-edited
scripts, retrieving information about \fsm{}s, drawing \fsm{}s, etc.
For example, numbered iteration could be implemented by defining the
following functions (though \Kleene{} already provides the convenient
\verb!{!\emph{n}\verb!}! postfix operator):

\begin{Verbatim}
$^iterate_by_recursion($fsm, #count) {
    if (#count > 0) {
        return $fsm $^iterate_by_recursion($fsm, #count - 1) ;
    } else {
        return "" ;
    }
}

$^iterate_by_loop($fsm, #count) {
    $result = "" ;
    while (#count > 0) {
        $result = $result $fsm ;
        #count = #count - 1 ;
    }
    return $result ;
}
\end{Verbatim}

\section{Input/Output}

\subsection{Scripts}

When Kleene is invoked ``bare'' from the command line, using

\begin{Verbatim}
$ java -jar Kleene.jar
\end{Verbatim}

\noindent
the \acro{gui} is launched, allowing users to type individual statements
for immediate intepretation.\footnote{The dollar sign in these examples
	represents the command-line prompt.  Kleene will initially be
	packaged for distribution as an executable \acro{jar} file.} A Kleene
	\acro{script} is a sequence of Kleene statements pre-edited and
	stored in a file.  Scripts can be executed by including their
	filenames in the command line,

\begin{Verbatim}
$ java -jar Kleene.jar myscript
$ java -jar Kleene.jar myscript1 myscript2 ...
\end{Verbatim}

\noindent
and Kleene will, by default, exit after all the scripts are executed.
The script names can be paths such as
\texttt{/Users/beesley/kleene/scripts/myscript.kl}.  By default, Kleene
(and Java programs in general) will assume that the script file is in the
default encoding of the operating system, and will convert it to Unicode
accordingly.\footnote{Java's String objects are always Unicode.  Text
input to Java programs is always converted to Unicode, one way or
another.  By default, text output from Java is converted from Unicode to
the default encoding of the operating system.}  If a script is not in the
default encoding of the operating system, its encoding should be specified
explicitly using the \texttt{-encoding} flag, which should appear before
the filename(s):

\begin{Verbatim}
$ java -jar Kleene.jar -encoding UTF-8    myscript
$ java -jar Kleene.jar -encoding Latin-1  myscript ...
\end{Verbatim}

\noindent
To cause Kleene to execute one or more scripts---typically start-up
scripts---and then launch the \acro{gui} for interactive input, add the
\texttt{-gui} flag anywhere among the script names.

\begin{Verbatim}
$ java -jar Kleene.jar myStartupScript -gui
\end{Verbatim}

To execute a Kleene script from the \acro{gui}, i.e.\@ from the Kleene
language itself, use the \texttt{source} command, which requires one
Kleene-regular-expression argument denoting a single-string filename or path,
with an optional comma-separated second argument (also denoting a language of
a single string) indicating the encoding of the source file.  In Kleene, a
sequence of alphabetic letters such as \texttt{myscript} is of course such a
single-string regular expression; and double-quoted strings, also regular
expressions, are appropriate if the name or path contains special characters
like periods, slashes and hyphens.

\begin{Verbatim}
// invoke from the Kleene GUI
// By default, the encoding of the script is assumed to be the 
//    default encoding of the operating system
source "myscript.kl" ;
source "/Users/beesley/kleene/scripts/testscript.kl" ;
source "myscript.kl", "UTF-8" ; // specify the encoding
\end{Verbatim}

By default, Kleene will attempt to read the script file using the default
encoding of your operating system, as perceived by Java. In Java code,
the default encoding of the operating system is supposed to be returned by
\texttt{System.getProperty("file.encoding")}, and on Linux this seems to work.
However, Apple \init{os
x} annoyingly reports the default encoding as ``MacRoman'' even if you've reset your locale
otherwise.\footnote{In Apple \init{os x}, you can set your LANG
environment variable to something like \verb!en_US.UTF-8!, which causes
the \verb!locale! command to report \init{utf-8} encoding, but inside
Java, \texttt{System.getProperty("file.encoding")} still returns
MacRoman.  If you, like me, have set up your whole \init{os x} system for default
\init{utf-8} encoding, and (naturally) expect Java to perceive the default encoding
as \init{utf-8},
you have to launch Kleene with something like \texttt{java -jar -Dfile.encoding-UTF-8
Kleene.jar}.}

In the Kleene \acro{gui}, the terminal widget has a pull-down \verb!Source!
menu that brings up a window that allows you to browse for the source file and
to indicate the encoding.  The encoding will be pre-set to the default encoding
of the operating system, but it can be changed if necessary.

\subsection{XML Input/Output}

An \fsm{} can be written to file in Kleene's 
\acro{xml} language using the \texttt{writeXml} command,
which takes as arguments a regular expression representing the \fsm{} to be
written, and a second regular expression indicating the file-path:

\begin{Verbatim}
writeXml $fsm, "/Users/beesley/kleene/xml/fsm.xml" ;
writeXml (dog|cat|elephant)s? , 
         "/Users/beesley/kleene/xml/animals.xml" ;
\end{Verbatim}

By default, the file is written in Unicode \acro{utf-8}, which is the
official default encoding for all \acro{xml} files.\footnote{The
\acro{xml} header of the file will be written as \texttt{<?xml
version="1.0" encoding="UTF-8"?>}, though the overt specification of
\acro{utf-8} encoding is redundant.}  Other encodings can be specified in
an optional third argument, but the only other recommendable encoding is
\acro{utf-16}.  \acro{xml} parsers are required to handle only
\acro{utf-8} and \acro{utf-16}, though they may handle other encodings.
Most users will want to stay with the default \acro{utf-8} encoding.

\begin{Verbatim}
// default UTF-8 output
writeXml $fsm, "/Users/beesley/kleene/xml/fsm.xml" ;

// explicit UTF-8 output
writeXml $fsm, "/Users/beesley/kleene/xml/fsm.xml, "UTF-8" ;

// explicit UTF-16 output
writeXml $fsm, "/Users/beesley/kleene/xml/fsm.xml", "UTF-16" ;
\end{Verbatim}

In the Kleene \acro{gui}, the \verb!writeXml! command can be invoked by
right-clicking on an \fsm{} icon and selecting the \verb!writeXml!
item.  The file selection window allows you to select the encoding,
which is pre-set to the recommended \acro{utf-8}.

The built-in function \verb!$^readXml()! reads from a Kleene \acro{xml} file and returns an
\fsm{}.

\begin{Verbatim}
// to read file fsm.xml in the current directory
$newfsm = $^readXml("fsm.xml") ;

// a full pathname can be specified
$newnfsm = $^readXml("/Users/beesley/kleene/xml/fsm.xml") ;
\end{Verbatim}

The encoding of an
\acro{xml} file is always 1) specified by an overt \texttt{encoding} in
the \acro{xml} header, if present, or 2) detectable automatically from the
\acro{bom} (Byte Order Mark), if present, or 3) \acro{utf-8} by default.
The reading of \acro{xml} files is now done using the \acro{rome} library's
\texttt{XmlReader} class, which knows how to detect the encoding of an
\acro{xml} file.

\subsection{\acro{dot} Output}

\fsm{}s can be written to file in the GraphViz \verb!dot!
format\footnote{\url{http://www.graphviz.org}} using the
\texttt{writeDot} command, which is similar to the \texttt{writeXml} command.

\begin{Verbatim}
writeDot $fsm, "/Users/beesley/kleene/xml/fsm.dot" ;
writeDot (dog|cat|elephant)s? , 
         "/Users/beesley/kleene/xml/animals.dot" ;
\end{Verbatim}

By default, the file will be written to file in the \acro{utf-8}
encoding.  A different encoding can be specified in an optional third
argument, but the only valid alternative is \acro{iso-8859-1}, because
the \verb!dot! parser can handle only \acro{utf-8} and \acro{iso-8859-1}.
Most users will want to stay with the default \acro{utf-8} encoding,
which is the only alternative if the sigma of the \fsm{} contains
Unicode symbols beyond the \acro{iso-8859-1} range.

In the Kleene \acro{gui}, the \verb!writeDot! command can be invoked by
right-clicking on an \fsm{} icon and selecting the \verb!writeDot! item.
The file selection window allows you to select the encoding, which is pre-set to the recommended \acro{utf-8}.  \acro{dot} output is used in the
background when you invoke the \texttt{draw} command, which can also be
generated by right-clicking on the \fsm{}'s icon and selecting the \texttt{draw} item, or
by double-clicking on the icon.

\subsection{Interactive Testing in the \acro{gui}}

For interactive manual testing of an \fsm{} within the \acro{gui},
use the Kleene \texttt{test} command, e.g.

\begin{Verbatim}
test $myfsm ;
test (dog|cat|rat) '[Noun]':"" ( '[Sg]':"" | '[Pl]':s ) ;
\end{Verbatim}

\noindent
The \texttt{test} command causes the indicated \fsm{} to be scanned for
multi-character symbols, which are used to build 
string-to-symbol tokenizers, one for the input side, and one for
the output side.\footnote{These tokenizers are currently
built using the Transliterator object from \acro{icu} (International
Components for Unicode).}  A testing window is opened, with string
input fields allowing the user to type in strings either for
analysis or generation.  An input string for analysis is tokenized
into symbols, built-in a one-string \fsm{}, and composed on the
output (lower) side of the transducer; the input (upper) side of
the composed \fsm{} is then extracted as the result.  Generation
is the same, except that the one-string \fsm{} is composed on the
input (upper) side of the original \fsm{}, and the output (lower)
side of the composed \fsm{} is extracted as the result.  If the
result language is reasonably finite, the individual strings are
displayed in the terminal window.

When the \fsm{} being tested contains \acro{other} (unknown) characters,
the output may be unintuitive; see Appendix~\ref{app:test} for more
details.

When an \fsm{} is finite (no cycles) relatively small and an acceptor, you can print out
the strings using the \texttt{print} command:

\begin{Verbatim}
$net = dog | cat | elephant ;
print $net ;
// output:
//    cat: 0.0
//    dog: 0.0
//    elephant: 0.0
\end{Verbatim}

[KRB:  Random upper and lower strings.  New functionality 2013-04.  Not well tested yet.]
For transducers and large \fst{}, even those with cycles, you can print out a random
selection of upper-side strings, or lower-side strings, using the \texttt{randInput} (alias
\texttt{randUpper}, \texttt{rinput} and \texttt{rupper}) command:

\begin{Verbatim}
randInput $fst ;
randUpper $fst ;
\end{Verbatim}

\noindent
Similarly, you can use the \texttt{randOutput} (alias \texttt{randLower}, \texttt{routput}
and \texttt{rlower}) command to print out a random selection of the output/lowerside
strings.

\begin{Verbatim}
randOutput $fst ;
randLower $fst ;
\end{Verbatim}

\noindent
These commands are often useful during development to ``see what the upper (or lower)
strings look like.''  These commands also have optional second and third arguments, the second
being the number of strings to display (the default is currently 15), and the third is the
maximum length of a string (default 50), which can be useful to avoid getting into infinite loops).

\begin{Verbatim}
// print 20 random input strings
randInput $fst, 20 ;

// print 10 random input strings, max 25 symbols
randUpper $fst, 10, 25 ;
\end{Verbatim}

\noindent
The ``random'' function are based on the OpenFst RandGen() function.

\section{Memory Management}

Memory management commands are intended for use by the developers and
maintainers of Kleene, not for typical users.

Kleene is based on the OpenFst library, and the finite-state machines 
(the actual states and arcs) are built by OpenFst and are therefore
\CPP{} objects, i.e.\@ all the states and arcs, comprising a \CPP{}
\fsm{} object, reside in the \CPP{} memory space.

On the Java side of Kleene, each finite-state machine is represented by a
Java Fst object,\footnote{Defined in the file \texttt{Fst.java}.} and
each allocated Fst object holds a pointer to the corresponding \CPP{}
finite-state machine object.\footnote{The Java Fst object also keeps
track of each \fsm{}'s private sigma, which is a concept alien to
OpenFst.}  The memory taken up by the Fst object itself is Java memory,
and can be reclaimed by the normal Java garbage collection.  But the
memory required by the states and arcs is \CPP{} memory and is invisible
to the Java garbage collector.

\subsection{Garbage Collection}

The Kleene parser and interpreter are written in Java, which
automatically performs garbage collection of Java objects when they no
longer have any references to them.  In general, Kleene users should
never have to worry about garbage collection, and even lower-level Java
programmers have little control over when garbage collection is done.  

In case it should become necessary or useful, Kleene offers the
\texttt{gc} command, which when evaluated suggests to Java that ``now
would be a good time to perform garbage collection''.\footnote{Java does
not necessarily have to accept the hint to perform garbage collection.
The \texttt{gc} command calls \texttt{System.gc()} and
\texttt{System.runFinalization()}, multiple times, which is alleged to
nudge Java more aggressively.  In a Java Fst object, the
\texttt{finalize()} method calls a native \CPP{} function that in turn
calls the normal \texttt{delete} command of \CPP{} to delete the \CPP{}
finite-state machine (the states and arcs in \CPP{} memory). The
\texttt{finalize()} method is advertised as the way to release/delete
``native'', i.e.\@ non-Java, resources, so it's just what Kleene needs;
but there is much debate among Java experts about if and how
\texttt{finalize()} should be relied on.}

\begin{samepage}
\begin{Verbatim}
$foo = foo ;
$bar = bar ;
delete $foo, $bar ;
gc ;
\end{Verbatim}
\end{samepage}

\subsection{Java Memory Usage}

Kleene also offers the \texttt{memory}
command, which prints out a summary showing the maximum memory, the
memory in use, and the free memory, as seen by Java.

\begin{Verbatim}
memory ;
\end{Verbatim}

\noindent
The format of the output, not shown here, will no doubt evolve as needs
become clear.  Note that Kleene currently has no way to interrogate the
memory usage or availability in the \CPP{} memory space.

\subsection{Java Fst Objects}

Kleene now keeps track of how many Java Fst objects have been allocated,
and how many have been ``finalized'', and this information can be
displayed by the \texttt{fsts} command.  When an Fst object has been
finalized, the Java object itself should be garbage collected, and the
corresponding OpenFst finite-state machine should be deleted.

\begin{Verbatim}
fsts ;
\end{Verbatim}

\noindent
The command also displays the number of ``live'' Fsts, which is the
difference between the number of allocated Fsts and the number of finalized Fsts.

\subsection{Contents of the Main Symbol Table}

The \texttt{symtab} command prints a list of the items defined in the main
``program'' symbol table, and these items should match the icons in the \acro{gui} symtab
window.

\begin{Verbatim}
symtab ;
\end{Verbatim}

\subsubsection{Contents of the Global Symbol Table}

The \texttt{gsymtab} command prints a list of the items defined in the global symbol
table, which is the root of the Kleene environment and the mother of the
``program'' symbol table.  The items in the list should correspond to the definitions
in the system-supplied \texttt{.kleene/global/predefined.kl} start-up script.

\begin{Verbatim}
gsymtab ;
\end{Verbatim}

\subsection{C++ Memory Space}

The Kleene system includes one custom-written non-OpenFst \CPP{} file,
named \texttt{kleeneopenfst.cc}, that defines the ``native'' (\CPP{})
functions directly callable from the Kleene interpreter and serves as the
\acro{jni} bridge between Kleene and the OpenFst library.  As with any
program involving C or \CPP{}, there is a significant danger of memory
leaks; finding and plugging them will be a high priority.

Another possible and significant source of \CPP{} memory leaks would be
the creation of native finite-state machines that are never deleted; the
\texttt{fsts} command, described above, may provide clues to tracking
down such leaks.  Kleene currently has no direct way to interrogate the
memory usage/availability in the \CPP{} memory space.\footnote{Memory
profilers may provide some help here.}


