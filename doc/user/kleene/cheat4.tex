
\chapter{Pre-defined Diacritic Functions}

The following functions are provided to make a network
diacritic-insensitive, which is significantly different from the
case-insensitivity previously described.  In particular, diacritic
insensitivity is unidirectional, modifying a network that already accepts
accented strings, e.g.\@ \emph{\'ecole}, to also accept partially or
completely unaccented versions of the same string, here \emph{ecole}.  If
the original network accepts only \emph{\'el\`eve}, then a
diacritic-insensitive version of the network would also accept
\emph{eleve}, \emph{\'eleve} and \emph{el\`eve}.  Note that if a network
accepts only the string \emph{resume}, applying diacritic insensitivity
will not cause it to accept strings like \emph{r\'esum\'e},
\emph{r\'esume}, \emph{resum\'e}, \emph{r\`esum\`e}, \emph{r\^es\"um\`e}
or any other strings with added or changed diacritics.\footnote{Modifying
a network to accept added or changed diacritics would, even if limited to
pre-composed Unicode characters, cause a network to explode in size.  If
the open-ended system of Combining Diacritical Marks were also handled,
the size of the network would be infinite.}

\vspace{0.5cm}

\noindent
\begin{tabular}{|l|l|}
\hline
\verb/$^opt_di($fst,/\verb!$proj="both"!\verb!)! & allow/add unaccented paths\\
\verb/$^opt_di!($fst,/\verb!$proj="both"!\verb!)! & allow/add unaccented paths\\
\hline
\end{tabular}

\vspace{0.5cm}

\noindent
For convenience, the following shorter aliases are pre-defined.

\vspace{0.5cm}

\noindent
\begin{tabular}{|l|l|}
\hline
\verb/$^di($fst,/\verb!$proj="both"!\verb!)! & allow/add unaccented paths\\
\verb/$^di!($fst,/\verb!$proj="both"!\verb!)! & allow/add unaccented paths\\
\hline
\end{tabular}

\vspace{0.5cm}

Diacritic insensitivity works for all accented---also known as
pre-composed---Unicode letter-characters, and it also includes
handling of Unicode Combining Diacritical Marks.\footnote{If a network
contains an arc labeled \emph{\'e}, diacritic insensitivity adds a
parallel arc labeled \emph{e}.  The unaccented version of the accented or
``pre-composed'' character is found via Unicode normalization to the
canonical decomposed \acro{nfd} form.  For an arc labeled with a
Combining Diacritical Mark, such as the \acro{combining acute accent},
diacritic insensitivity adds a parallel epsilon arc.}  Thus if a network
originally accepts just \emph{\'ecole}, the diacritic-insensitive version
will accept \emph{\'ecole} and \emph{ecole} whether the original
\emph{\'e} was encoded as the single Unicode character U+00E9
(\acro{latin small letter e with acute}) or as the sequence of characters
U+0065 (\acro{latin small letter e}) followed by U+0301 (\acro{combining
acute accent}).\footnote{OpenFst networks created by Kleene always store
one Unicode character (using its standard code point value) in one arc
label, and this includes supplementary Unicode characters and Combining
Diacritical Marks.}


